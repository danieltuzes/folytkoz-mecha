\documentclass[a4paper]{article}

%% Language and font encodings
\usepackage[english]{babel}
\usepackage[utf8x]{inputenc}
\usepackage[T1]{fontenc}

%% Sets page size and margins
\usepackage[a4paper,top=3cm,bottom=2cm,left=3cm,right=3cm,marginparwidth=1.75cm]{geometry}

%% Useful packages
\usepackage{amsmath}
\usepackage{mathabx}
\usepackage{graphicx}
\usepackage[colorinlistoftodos]{todonotes}
\usepackage[colorlinks=true, allcolors=blue]{hyperref}

\title{%
  6.tétel \\
  \large Hidrosztatika}

\date{}
\begin{document}
\maketitle

Mindeddig szilárd anyagok deformációjával és változásával foglalkoztunk, azonban a való életben nagyon sokszor kerülünk kapcsolatba folyadékokkal és gázokkal is. Fontos tudni hogyan hat a folyó vize a híd pilléreire, vagy épp a hajó testére de már az is érdekes lehet, hogy a lufi oldalát hogyan nyomja a benne levő gáz.
\\
\\
A szilárd anyagoknál a deformációra bevezettünk egy $\underline{u}(\underline{r})$ vektorteret. Ez a folyadékoknál már nem egy célszerű leírási mód hiszen folyadékok vagy gázok áramlásánál nem tudunk olyan hatékonyan egy kitüntetett pontot vizsgálni, amihez képest vizsgáljuk a pontok elmozdulását.
\\
Jobban érdekel minket a következő két dolog: a $\underline{v}(\underline{r},t)$ sebességtér és a $\rho(\underline{r},t)$ sűrűségmező. Mostantól erre a két mennyiségre figyeljünk.
\\
\\
\section{Pascal-törvény}

A következőben vegyünk egy $\Delta A$ felületet, amire ható erő egy folyadék nyomásából ered. Mivel:
\begin{equation}
\Delta F=-p\Delta A
\end{equation}
és
\begin{equation}
F=\sigma A
\end{equation}
ezért:
\begin{equation}
\sigma=\begin{pmatrix}
    -p & 0 & 0 \\
    0 & -p & 0 \\
    0 & 0 & -p
\end{pmatrix}=-p\hat{I}
\end{equation}
\\
Az, hogy a mátrix diagonális, azt jelenti, hogy a folyadékban nincsenek kitüntetett irányok. 
\\
\\
Egy felületre ható erő (ami folyadék nyomásából származik) párhuzamos a felület normálisával és arányos a felület nagyságával.
\\
Vizsgáljuk az egyensúly esetét. Az egyensúly feltétele:
\begin{equation}
div\sigma+\underline{f}=0
\end{equation}
Ebből:
\begin{equation}
div\sigma=div\begin{pmatrix}
    -p & 0 & 0 \\
    0 & -p & 0 \\
    0 & 0 & -p
\end{pmatrix}=-gradp
\end{equation}
Tehát:
\begin{equation}
\underline{f}-gradp=0
\end{equation}
\\
Ha gravitációs erőtérben nézzük, akkor $\underline{f}=\rho\underline{g}=-\rho grad\Phi$, ahol $\Phi$ a gravitációs potenciál. Így:
\begin{equation}
gradp+\rho grad\Phi=0
\end{equation}
\\
Ha $\rho$ állandó:
\begin{equation}
grad(p+\rho\Phi)=0
\end{equation}
\\
Tehát $p+\rho\Phi$ állandó. Ez azt jelenti, hogy az azonos nyomású és azonos potenciálú helyek egybeesnek. A gravitációs erőtérben a potenciál $gh$, innen láthatjuk, hogy
\begin{equation}
p=p_{0}+\rho gh
\end{equation}
\\
Ahol $p_{0}$ a $h=0$-ban mért nyomás.
\\
\\
\section{Felhajtóerő}
Vizsgáljuk egy folyadékba merülő téglatestre ható erőket! A folyadék felszínével párhuzamos oldalak területe legyen $a\cdot a$, a harmadik él pedig legyen $b$. A téglatest felszíntől vett távolságát hívjuk $h$-nak.
A nyomásból származó erő, a folyadék sűrűségének ismeretében:
\begin{equation}
F_{p}=p(z_{1})a^{2}-p(z_2)a^2=(\rho gh-\rho g(h+b))a^2=-\rho gba^2=-\rho gV
\end{equation}
Ezt nevezzük felhajtóerőnek.
\\
Szabálytalan test esetén:
\begin{equation}
F_f=\oint\hat{\sigma}dA=\int div\hat{\sigma}dV=-\int\rho gdV=-\rho gV
\end{equation}
Így látahatjuk hogy a test alakjától nam, csak a térfogatától függ a felhajtóerő.
\\
\\
\section{Forgó folyadék}
Forgassunk egy folyadékot és vizsgáljuk meg milyen alakot fog felvenni.
Az egyensúly feltétele a következő:

\begin{equation}
\underline{f}-gradp=0
\end{equation}
\begin{equation}
\underline{f}=\rho g+\rho\omega s
\end{equation}
\\
Ahol $s=(x;y;0)$. A centrifugális erősűrűség ($\rho\omega s$) előáll $\Phi_c$ negatív gradienseként amennyiben $\Phi_c=-\frac{1}{2}\rho\omega^2(x^2+y^2)$ Így a potenciális energiasűrűség $\Phi=\rho gz-\frac{1}{2}\rho\omega^2(x^2+y^2)$. Tehát:
\begin{equation}
grad(p+\rho gz-\frac{1}{2}\rho\omega^2(x^2+y^2))=0 \rightarrow p+\rho gz-\frac{1}{2}\rho\omega^2(x^2+y^2)=p_0
\end{equation}
\\
Ha megfelelően választjuk a koordinátarendszert, $p=p_0$ így amit kapunk
\begin{equation}
\rho gz=\frac{1}{2}\rho\omega^2(x^2+y^2)
\end{equation}
\\
Ami egy forgási paraboloid képlete. Ezzel megkaptuk a forgó folyadák felszínének alakját.
\\
\section{Barometrikus magasságformula}

Gravitációs erőtérben az egyensúly feltétele: $-gradp-\rho grad\Phi=0$. Megfigyelhetjük, hogy gázok esetén a hely függvényében $\rho$ nem lesz állandó. Ideális gázok esetén:
\begin{equation}
pV=\frac{m}{M}RT
\end{equation}
\begin{equation}
\frac{p}{\rho}=\frac{1}{M}RT\rightarrow \rho=\frac{Mp}{RT}
\end{equation}
\\
Használjunk izoterm közelítést, azaz tekintsük állandónak a hőmérsékletet.
\\
\begin{equation}
gradp+\frac{M}{RT}pgrad\Phi=0
\end{equation}
\begin{equation}
grad\Phi=-g
\end{equation}
\begin{equation}
gradp-\frac{M}{RT}pg=0
\end{equation}
\\
Csak z irányban változik a nyomás: $\frac{dp}{dz}+\frac{Mg}{RT}p=0$. Ezt megoldva megkapjuk, hogy
\begin{equation}
p=p_0+e^{-\frac{Mgz}{RT}}
\end{equation}
\\
Mivel a sűrűség arányos a nyomással:
\begin{equation}
\rho=\rho_0+e^{-\frac{Mgz}{RT}}
\end{equation}
\\
Tehát a nyomás és a sűrűség is exponenciálisan csökken a magassággal. Ha az $M=mA$ és az $\frac{R}{A}=k_b$ jelölést alkalmazzuk ($A$ az Avogadro-szám, $k_b$ pedig a Boltzmann-állandó):
\begin{equation}
\rho=\rho_0+e^{-\frac{mgz}{k_bT}}
\end{equation}
\\
Ezt nevezzük barometrikus magasságformulának



\end{document}