\documentclass[a4paper]{article}

%% Language and font encodings
\usepackage[english]{babel}
\usepackage[utf8x]{inputenc}
\usepackage[T1]{fontenc}

%% Sets page size and margins
\usepackage[a4paper,top=3cm,bottom=2cm,left=3cm,right=3cm,marginparwidth=1.75cm]{geometry}

%% Useful packages
\usepackage{amsmath}
\usepackage{mathabx}
\usepackage{graphicx}
\usepackage[colorinlistoftodos]{todonotes}
\usepackage[colorlinks=true, allcolors=blue]{hyperref}

\title{%
  5.tétel \\
  \large Csavarás, lehajlás}

\date{}
\begin{document}
\maketitle

\section{Csavarás}

Sokszor fordul elő, hogy bizonyos testek megcsavarodnak. Ezt a fajta deformációt vizsgáljuk a következőkben:
\\
\\
A csavarási deformációt vizsgáljuk egy hengeren, melynek alja rögzítve van. A henger legyen $l$ magasságú és $R$ sugarú. Tetejét csavarjuk el $\Delta\varphi$ szögógel!
\\
Vizsgáljunk egy pontpárt a hengeren. A két pontot jelöljük ki úgy, hogy eredetileg egymás alatt/felett voltak a henger két alapjának a szélén. Legyen a felső pont eredeti pozíciója $A$, az alsó pontté $B$. Csavarás után az alsó pont marad $B$-ben azonban a felső pont átkerül $A'$-be. Az $ABA'\sphericalangle$-et jelöljük $\gamma$-val.
\\
Ezeket a jelöléseket felhasználva adódik a következő egyenlet:

\begin{equation}
	\gamma(r)=\Delta\varphi\frac{r}{l}
\end{equation}
\\
\\
Mivel a csavarás a tiszta nyírás egy speciális fajtája (térfogatváltozás nélküli deformáció), ezért alkalmazható rá a nyírásoknál tanult egyenlet: 

\begin{equation}
\tau=\mu\gamma
\end{equation}

\begin{equation}
\tau=\mu\frac{\Delta\varphi}{l}r
\end{equation}
\\
Most nézzük meg a hengerre ható forgatónyomatékot. Mivel tudjuk, hogy a forgatónyomaték függ az erőkar nagyságától ($M=F\cdot r$), ezért láthatjuk, hogy a tengelytől azonos távolságokban a forgatónyomaték ugyan annyi. Vegyük tehát a henger keresztmetszetét és osszuk fel koncentrikus körgyűrűkre, melyeknek sugárirányú kiterjedése $\Delta r\rightarrow 0$. Egy körgyűrű területe így $2\pi r\Delta r$. Mivel $\tau=\frac{F}{A}$ ezért a forgatónyomatékot a következő módon kaphatjuk meg:

\begin{equation}
M=\int^{A}_{0}\tau(r)rdA=\int^{R}_{0}\tau(r)r2\pi rdr=\int^{R}_{0}\mu\frac{\Delta\varphi}{l}2r^{3}\pi dr=\frac{\pi\mu}{2l}R^{4}\Delta\varphi
\end{equation}
\\
Tehát a forgatónyomaték arányos az elcsavarás szögével, a henger sugarának negyedik hatványával, valamint fordítottan arányos a henger magasságával.

\newpage
\section{Elhajlás}
Az elhajlás a való életben talán még gyakrabban előforduló deformációtípus mint a csavarás. Vizsgáljuk meg ezt is egy példán keresztül!
\\
\\
Rögzítsünk egy függőleges falhoz egyik lapjánál egy téglalap alapú hasábot. A gravitációs erő hatására a hasáb lehajlik, minél távolabb nézzük a faltól, annál nagyobb mértékben. A testnek lesz egy olyan része ami nem nyúlik meg, ezt nevezzük neutrális tartománynak. Illesszünk ehhez a neutrális tartományhoz egy
$R$ sugarú érintő kört, amelynek illeszkedési szöge $\Delta\varphi$. Mivel itt nincs deformáció, ezért tudjuk, hogy $\varepsilon (R)=0$. $R$-től $\xi$ távolságban:

\begin{equation}
\varepsilon (\xi)=\frac{\Delta\varphi (R+\xi)-\Delta\varphi R}{\Delta\varphi R}=\frac{\xi}{R}
\end{equation}
\\
A feszültség:
\begin{equation}
\sigma = E\varepsilon=E\frac{\xi}{R}
\end{equation}
\\
Newton II. törvénye alapján tudjuk hogy az eredő erő 0 lesz egyensúlyban. Mivel a neutrális tartomány mellett a belső erők elletétes irányúak, ezért ezen a tartományon az eredő erő vízszintes komponense 0. Tehát:

\begin{equation}
F=\int\sigma dA=0
\end{equation}
\\
Ebből az látszik, hogy a semleges tartomány a súlypontban helyezkedik el.
\\
Az összforgatónyomaték is 0 kell hogy legyen az egyensúly beálltához. A fedőlap vízszintes oldalának hosszúsága legyen $a$, magassága pedig $b$ Adjuk meg a forgatónyomatékot a neutrális zónától $\xi$ távolságra:
\begin{equation}
M=\int\sigma\xi dA
\end{equation}
\\
$dA$-t átírhatjuk $ad\xi$-re, mivel $a$ konstans és a neutrális felülettől $\xi$ távolságban megegyeznek az erők. Így  a következő integrállá alakíthatjuk:
\begin{equation}
M=\int_{-\frac{b}{2}}^{\frac{b}{2}}E\frac{\xi}{R}\xi ad\xi=\frac{E}{R} I
\end{equation}
\\
Itt $I$ egy általunk definiált új érték, amit hajlítási nyomatéknak nevezünk.
\begin{equation}
I=\int\xi^2ad\xi
\end{equation}
\begin{equation}
\frac{E}{R}a\bigg[\frac{\xi^3}{12}\bigg]_{-\frac{b}{2}}^{\frac{b}{2}}=\frac{E}{R}\frac{ab^3}{12}
\end{equation}
\\
A külső és belső erők forgatónyomatékának összege 0: $\frac{E}{R}I-F(l-x)=0$. Legyen a neutrális felület simulókörének középpntja $(x_0,0)$. A kör egyenlete így: $(x-x_0)^2+y^2=R^2\rightarrow y=\pm\sqrt[]{R^2-(x-x_0)^2}$
\begin{equation}
\frac{dy}{dx}=\pm\frac{1}{\sqrt[]{R^2-(x-x_0)^2}}(-2(x-x_0))
\end{equation}
\\
Ha csak $x_0$ környezetét vizsgáljuk ($x\approx x_0$):
\begin{equation}
\frac{d^2y}{dx^2}=\pm\frac{1}{R}
\end{equation}
Forgatónyomateékra vonatkozó egyenlet:
\begin{equation}
EI\frac{d^2y}{dx^2}+F(l-x)=0
\end{equation}
A határfeltételek a következők: $y(0)=0, \frac{dy}{dx}(0)=0$
Sejtés: keressük a megoldást $a+bx+\frac{c}{2}x^2+\frac{d}{4}x^3$ alakban.
\\
$y(0)=0\rightarrow a=0$ és $\frac{dy}{dx}(0)=0\rightarrow b=0$
\begin{equation}
\frac{dy}{dx}=cx+\frac{d}{2}x^2
\end{equation}
\begin{equation}
\frac{d^2y}{dx^2}=c+dx
\end{equation}
\begin{equation}
EI(c+dx)+F(l-x)=0
\end{equation}
\begin{equation}
EIc=-Fl
\end{equation}
\begin{equation}
EId=F
\end{equation}
Tehát:
\begin{equation}
c=-\frac{Fl}{IE}
\end{equation}
\begin{equation}
d=\frac{F}{IE}
\end{equation}
Tehát a lehajlás egyenlete:
\begin{equation}
y(x)=\frac{F}{EI}\Big(\frac{1}{6}x^3-\frac{l}{2}x^2\Big)
\end{equation}

\end{document}