\documentclass[a4paper, titlepage]{article}
\usepackage[utf8]{inputenc}
\usepackage[T1]{fontenc}
\usepackage[magyar]{babel}
\usepackage{graphicx}
\usepackage{geometry}
\usepackage{amsmath,amssymb}
\geometry{margin=1.2in}
\begin{document}
\begin{titlepage}
	\title{\textbf{Folytonos közegek mechanikája}\\
	Tételkidolgozás\\
	10. tétel\\
	\textit{Hang terjedése gázban, hullámegyenlet}}
	\author{Készítette Pesznyák Dávid\\
	Dr. Groma István előadásai alapján (2018)}
	\date{}
\end{titlepage}
\maketitle
\newpage
\subsection*{A hang mint nyomáshullám}
A hangot nyomáshullámként értelmezhetjük, ahogy gázban vagy folyadékban halad. Éppen emiatt nem hanyagolhatjuk el ebben az esetben a közeg összenyomhatóságát. Összenyomhatatlan gázban vagy folyadékban a hang sebessége végtelennek adódna, ami közel sem felel meg a valóságnak.\\\\
A probléma további vizsgálatához felhasználhatjuk a mozgásegyenletet és a kontinuitási egyenletet:
\begin{equation}
\rho\frac{\partial \vec{v}}{\partial t}\;+\;\rho(\vec{v}\;\vec{\nabla})\vec{v}=-grad(p),
\end{equation}
ahol $\rho$ a közeg sűrűsége, $\vec{v}$ az áramlási sebességet leíró vektor, $p$ a nyomás és $t$ pedig az időt jelöli. A mozgásegyenletből most elhagyjuk a súrlódási erőt tartalmazó tagokat.
\begin{equation}
\frac{\partial \rho}{\partial t}\;+\;div(\rho \vec{v})=0
\end{equation}
A kontinuitási egyenlet pedig továbbra is az anyagmegmaradást biztosítja. Ezen felül még feltételezünk valamilyen ismert kapcsolatot nyomás és sűrűség között ($p(\rho)$).\\\\
A mozgásegyenlet továbbra sem lineáris a $(\vec{v}\;\vec{\nabla})\vec{v}$ tag miatt, így az egyenlet megoldása továbbra sem egyszerű.
\subsection*{A hullámegyenlet}
Triviális megoldásként adódik speciális esetben, ha a közeg áll, és a nyomás- és sűrűségértékek is időfüggetlen konstansok:
$$
\vec{v}=\vec{0}\qquad p=p_0\qquad\rho=\rho_0
$$
Ha megperturbáljuk a rendszert, és feltételezzük, hogy az is megoldás, akkor a következő mennyiségeket írhatjuk majd vissza az egyenleteinkbe:
$$
\vec{v}=\vec{0}\;+\;\vec{v}\qquad p(\vec{r},t)=p_0\;+\;\delta p(\vec{r},t)\qquad\rho(\vec{r},t)=\rho_0\;+\;\delta\rho(\vec{r},t)
$$
Visszaíráskor elhanyagolhatjuk a $(\vec{v}\;\vec{\nabla})\vec{v}$-s tagot, mivel a $\vec{v}$ perturbáció mértéke kicsi, így azt négyzetre emelve már az egész tag elhagyható, és a $\delta$-ban négyzetes tagokat is elhagyhatjuk hasonló indokok miatt. Az új egyenleteink tehát:
\begin{equation}
\label{eq:cont}
\frac{\partial\delta\rho}{\partial t}\;+\;\rho_0 div(\vec{v})=0
\end{equation}
\begin{equation}
\label{eq:rot}
\rho_0\frac{\partial \vec{v}}{\partial t}=-grad(\delta p)
\end{equation}
Felhasználva a feltételezett és ismert $p(\rho)$ összefüggést \aref{eq:NS}. egyenlet tovább alakítható:
$$
\rho_0\frac{\partial \vec{v}}{\partial t}=-grad(\delta p)=-grad\bigg(\frac{dp}{d\rho}\bigg\vert_{\rho_0}\delta\rho\bigg)=-grad(c^2\delta\rho)=-c^2grad(\delta\rho),
$$
$\frac{dp}{d\rho}\big\vert_{\rho_0}=c^2$
\begin{equation}
\label{eq:NS}
\rho_0\frac{\partial \vec{v}}{\partial t}=-c^2grad(\delta \rho)
\end{equation}
Ezek után vegyük \aref{eq:cont}. egyenlet időderiváltját:
\begin{equation}
\label{eq:(1)}
\frac{\partial^2 \delta\rho}{\partial t^2}+\rho_0 div\bigg(\frac{\partial \vec{v}}{\partial t}\bigg)=0
\end{equation}
és \aref{eq:NS}. egyenlet divergenciáját:
\begin{equation}
\label{eq:(2)}
\rho_0 div\bigg(\frac{\partial \vec{v}}{\partial t}\bigg)=-c^2divgrad(\delta \rho)=-c^2\bigtriangleup \delta \rho
\end{equation}
\newpage
\noindent Észrevehető, hogy \aref{eq:(1)}. és \aref{eq:(2)}. egyenletben is megjelenik a $\rho_0 div\big(\frac{\partial \vec{v}}{\partial t}\big)$-s tag. A két egyenletet, ha kivonjuk egymásból és 0-ra rendezzük, akkor a következő összefüggéshez jutunk:
\begin{equation}
\frac{\partial^2 \delta\rho}{\partial t^2}-c^2\bigtriangleup\delta\rho=0
\end{equation}
Ugyanez felírható nyomás változóval is:
\begin{equation}
\frac{\partial^2 \delta p}{\partial t^2}-c^2\bigtriangleup\delta p=0
\end{equation}
Ezt nevezzük hullámegyenletnek, tehát a hang az valamilyen nyomás- vagy sűrűséghullám, és ez általánosan is felírható tetszőleges $\Phi$ terjedő mennyiségre:
$$
\frac{\partial^2 \Phi}{\partial t^2}-c^2\bigtriangleup\Phi=0
$$
Továbbá felismerhető, hogy ez az egyenlet a $c^2$-es szorzó nélkül egy homogén D'Alambert-egyenlet:
$$
\Box \Phi=0
$$
Visszatérve \aref{eq:rot}. egyenlethez, és véve mindkét oldalának rotációját, és felhasználva azt az azonosságot, hogy $rotgrad\Phi=0$, valamint a Young-tétel szerint felcserélhetőnek vesszük a parciális deriválások sorrendjét, a következő kifejezéshez jutunk:
\begin{equation}
\rho_0 \frac{\partial}{\partial t}(rot(\vec{v)})=\vec{0}
\end{equation}
Ebből látható, hogy a sebesség-vektormező rotációja időben állandó, tehát az áramlás örvényessége megmarad.
\subsection*{Az egyenlet megoldása}
Az egyenletek megoldása lehet egy síkhullám, amiről belátható, hogy longitudinálisnak kell lennie (a probléma gömbhullámként is kezelhető). A síkhullámot leíró összefüggések a következők:
\begin{equation}
\label{eq:long}
\delta\rho=\delta\rho_0 e^{i(\omega t+\vec{k}\vec{r})}\qquad \vec{v}=\vec{v}_0e^{i(\omega t+\vec{k}\vec{r})},
\end{equation}
ahol $\omega$ a körfrekvencia, $t$ idő, $\vec{k}$ a hullámszámvektor és $\vec{r}$ pedig helyvektor, és ezeknek skaláris szorzatát vesszük a kitevőben. Síkhullám esetén az azonos fázisú felületek egy síkban vannak. Ezeket behelyettesíthetjük \aref{eq:cont}. (kontinuitási) egyenletbe:
$$
i\omega\delta\rho_0 e^{i(\omega t+\vec{k}\vec{r})}\;+\;i\rho_0(\vec{v}_0\vec{k})e^{i(\omega t+\vec{k}\vec{r})}=0
$$
Legyeszerűsíthetünk az exponenciális tényezővel és az imaginárius egységgel:
$$
\delta \rho_0\omega\;+\;\rho_0(\vec{v}_0\vec{k})=0
$$
A sebességvektor és a hullámszámvektor skaláris szorzata miatt a sebességnek csak a $\vec{k}$-irányú vetülete számítható.\\\\
Ha \aref{eq:(1)}. egyenletbe is beírjuk a síkhullámot leíró összefüggéseket, akkor a következőkre jutunk:
$$
i\omega\rho_0 \vec{v}_0 e^{i(\omega t+\vec{k}\vec{r})}=-c^2i\vec{k}\delta\rho_0 e^{i(\omega t+\vec{k}\vec{r})}
$$
Ismét leegyszerűsíthetünk az exponenciális tényezővel és az imaginárius egységgel:
$$
\omega \rho_0 \vec{v}_0=-c^2\delta \rho_0 \vec{k}
$$
Látható, hogy az egyenlet két oldala csak akkor lehet egyenlő, ha $\vec{v}_0$ sebességvektor és $\vec{k}$ hullámszámvektor párhuzamos. Így mikor a sebesség $\vec{k}$-irányú vetületét számítjuk, akkor megkapjuk a teljes sebességvektort, mivel csak olyan irányú komponesei vannak. Így beláttuk, hogy a hullámnak longitudinálisnak kell lennie, mivel a hullám sebessége és terjedési iránya egybeesik.\newpage
\noindent Ha behelyettesítünk a kapott hullámegyenletbe a síkhullám mennyiségeivel, akkor a körfrekvenciára az alábbi összegüggést kaphatjuk:
$$
\frac{\partial^2\delta \rho}{\partial t^2}-c^2\bigtriangleup \delta \rho=-\omega^2\delta\rho+c^2(k_x^2+k_y^2+k_z^2)\delta\rho=-\omega^2\delta\rho+c^2|\vec{k}|^2\delta\rho=0
$$
Ezt tovább egyszerűsítve:
$$
\omega^2=c^2|\vec{k}|^2
$$
\subsection*{Hangsebesség meghatározása}
A hang sebességének meghatározásának feladatát kétfelé bonthatjuk: kis- és nagyfrekvenciás hullámokra.\\\\
Kis frekvencia esetén a folyamat lassú, a hőmérséklet állandó marad, így a termodinamikában ismert kifejezés szerint izotermnek nevezzük. Felírható, hogy:
\begin{equation}
\frac{p}{\rho}=\frac{R}{M}T,
\end{equation}
ahol az újonnan bevezett mennyiségek: $R$ az egyetemes gázállandó ($\approx 8.314 \frac{J}{mol\,K}$), $M$ a moláris tömeg és $T$ a hőmérséklet. Ekkor a már korábban meghatározott $c^2=\frac{dp}{d\rho}\big\vert_{\rho_0}$ összefüggés alapján:
\begin{equation}
c^2=\frac{R}{M}T
\end{equation}
Nagy frekvenicaérték esetén azt feltételezzük, hogy olyan gyors a folyamat, hogy nincs hőcsere, így adiabatikus közelítést alkalmazhatunk:
\begin{equation}
\frac{p}{\rho^\kappa}=const.
\end{equation}
Ezt $p$-re rendezve és véve mindkét oldal $\rho$ szerinti deriváltját:
$$
\frac{dp}{d\rho}=\kappa\cdot const.\cdot \rho^{\kappa-1}=\kappa\frac{p}{\rho}=\kappa\frac{R}{M}T
$$
Ismét felhasználva az előbbi összefüggést $c^2$-re:
\begin{equation}
c^2=\kappa\frac{R}{M}T
\end{equation}
Ez a két vizsgált folyamat határerestnek tekinthetó, és látható, hogy csak egy $\kappa$ faktorban térnek el egymástól. Mivel $\kappa=\frac{c_p}{c_v}$, ez a különbség sem nagy.\\\\
A kompresszibilitást ($\kappa_T$) vizsgálva is vonhatunk le következtetéseket. A kompresszibilitást a következőképpen számíthatjuk:
\begin{equation}
-\frac{1}{V}\frac{dV}{dp}=-\frac{1}{\rho}\frac{d\rho}{dp}=\kappa_T,
\end{equation}
ahol $V$ a térfogat. Ezt az egyenletet átalakítva a $c^2$ a következőnek adódik:
\begin{equation}
c^2=\frac{1}{\kappa_T \rho}
\end{equation}
Emiatt $c$ sebesség összenyomhatatlan közeg esetén végtelennek adódik, mivel ekkor $\kappa_T=0$. Valódi összenyomhatatlan közeg nem létezik, viszont azt tudjuk, hogy a folyadékok sokkal kevésbé összenyomhatók, mint a gázok, tehát folyékony közegben a hanghullámok is számotteveően gyosrabban terjednek.
$$
c_{folyadek}>>c_{gaz}
$$
\end{document}