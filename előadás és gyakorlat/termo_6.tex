\section{Folyamatok és hőtani alapmennyiségek}
\emph{Fajhő; Hőkapacitás; Nyílt folyamatok ideális gázokkal: izoterm, izochor, adiabatikus, politrop folyamatok; Robert-Mayer-egyenlet}

A belső energia és a termodinamika I. főtételének ismeretében definiálhatunk néhány hasznos fogalmat. kvázisztatikus folyamatokra az I. főtétel alakja:
\begin{align}
    \m dU = \delta W + \delta Q.
\end{align}
Legyen a:
\begin{align}
    \tn{hőkapacitás: }\mathcal C &= \frac{\delta Q}{\m dT}, &\tn{mértékegysége: } [\mathcal C] &= \frac{\tn J}{\tn K},\\
    \tn{mólhő: } C &= \rec n \frac{\delta Q}{\m dT},  &\tn{mértékegysége: } [C] &= \frac{\tn J}{\tn{mol}\cdot \tn K},\\
    \tn{fajhő: }c &= \rec m \frac{\delta Q}{\m dT}, &\tn{mértékegysége: } [c] &= \frac{\tn J}{\tn {kg}\cdot \tn K}.
\end{align}
A \emph{hőkapacitás} megadja, hogy mennyi hőt kell közölni az anyaggal, hogy 1K-nel megnöveljük a hőméréskletét. A \emph{mólhő} (moláris hőkapacitás) anyagmennyiségtől független, anyagra jellemző állandó, mely megadja, hogy mennyi hőt kell közölni 1 mólnyi anyaggal, hogy 1K-nel megnőveljük a hőmérsékletét. A \emph{fajhő} (fajlagos hőkapacitás) tömegtől független mennyiség, szintén az anyagra jellemző állandó, mely megadja, hogy mennyi hőt kell közölni 1kg anyaggal, hogy 1K-nel megnöveljük a hőmérsékletét. Látható, hogy célszerű lesz a továbbiakban inkább a mólhővel vagy a fajhővel számolnunk, ugyanis azok az anyagra jellemző mennyiségek.

\emph{Nyílt folyamatnak} nevezzük azokat a folyamatokat, melyek során a rendszer egy kezdőállapotból egy másik végállapotba kerül át. Ennek több fajtáját különböztetjük meg. Határozzuk most meg az állandó térfogatú, illetve az állandó nyomású folyamatra a belső energiát illetve a fajhőt ebben a két speciális esetben.

\emph{Izochor} folyamatok során a térfogat állandó. 
A belső energia a térfogat és a hőmérséklet függvénye, viszont $\m dV=0$, így ha megnézzük a belső energia teljes deriváltját, akkor:
\begin{align}\label{eq:izochor_U}
    \m dU = \underbrace{-\pres\m dV}_{=0}+\delta Q = \delta Q,\quad \m dU(V,T) = \underbrace{\pd UV\bigg|_T\m dV}_{=0} + \pd UT\bigg|_V\m dT \follows \delta Q = \pd UT\bigg|_V\m dT,
\end{align}
amiből az izochor folyamatokra jellemző mólhő megkapható:
\begin{align}
     C_V = \rec n \frac{\delta Q}{\m dT} = \rec n \pd UT\bigg|_V.
\end{align}
\emph{Izobár} folyamatokban a nyomás állandó. Ha a belső energiát most a nyomás és a hőmérséklet függvényében írjuk fel, akkor:
\begin{align}
    \m U = -\pres\m dV+\delta Q,\quad \m dU(p,T) = \underbrace{\pd Up\bigg|_T \m d\pres}_{=0} + \pd UT\bigg|_\pres \m dT \follows -\pres\m dV +\delta Q = \pd UT\bigg|_\pres \m dT.
\end{align}
A mólhő meghatározásához ki kellene fejeznünk $\m dV$-t $\m dT$-vel. Ehhez vizsgáljuk meg a következőt:
\begin{align}
    V(\pres,T) \follows \m dV = \underbrace{\pd V\pres\bigg|_T\m d\pres}_{=0} + \pd VT\bigg|_\pres \m dT \follows -\pres\pd VT\bigg|_\pres \m dT +\delta Q = \pd UT\bigg|_\pres \m dT,
\end{align}
amiből a mólhőre kapjuk \emph{izobár} folyamatok esetén, hogy:
\begin{align}
    C_\pres = \rec n\z{\pd UT\bigg|_\pres + \pres \pd VT\bigg|_\pres} = \rec n \frac{\partial(\overbrace{U+\pres V}^{\equiv H})}{\partial T}\bigg|_\pres =\rec n\pd HT\bigg|_\pres,
\end{align}
ahol bevezettük a $H=U+\pres V$ mennyiséget, amit \emph{entalpiának} nevezünk, később még részletesebben is foglalkozunk vele.

\emph{Ideális gázoknak} ismerjük a belső energiáját illetve állapotegyenletét\footnote{\,Emlékezzünk rá, hogy ezek:\begin{align}
    U(\pres,T) = \frac f2 nRT,\quad V(\pres,T) = \frac{nRT}{\pres}
\end{align}}, amiből az állandó térfogatú illetve állandó nyomású mólhő meghatározható:
\begin{align}\label{eq:cv_cp}
    C_V = \rec n \pd UT\bigg|_V = \frac f2 R,\quad C_\pres = \rec n \pd{(U+\pres V)}{T}\bigg|_\pres = \rec n\z{\frac f2 nR+ \pres\frac{nR}{\pres}} = \frac{f+2}{2}R.
\end{align} 
Az \emph{ideális gáz Robert Mayer\footnote{\,Julius Robert Mayer, 1814-1878.}-egyenletének} nevezzük a következő érdekes megállapítást:
\begin{align}
    C_\pres - C_V =R \follows c_\pres - c_V = \frac RM.
\end{align}
Határozzuk meg most általánosan is a Robert Mayer-egyenletet.
Ehhez írjuk fel a kétféle mólhő különbségét az eddigiek ismeretében:
\begin{align}
    C_\pres - C_V = \rec n\z{\pd UT\bigg|_\pres + \pres \pd VT\bigg|_\pres - \pd UV\bigg|_V}.
\end{align}
Ezt egyszerűbb alakra is lehet hozni, ha felhasználjuk az alábbit:
\begin{align}
    U\big(V(\pres,T),T \big) \follows \pd UT\bigg|_\pres = \pd UV\bigg|_T\cdot\pd VT\bigg|_\pres + \pd UT\bigg|_V,
\end{align}
ami segítségével a \emph{Robert Mayer-egyenlet}:
\begin{align}\label{eq:RM}
    C_\pres -C_V = \rec n\z{\pres\pd VT\bigg|_\pres + \pd UV\bigg|_V\cdot \pd TV\bigg|_\pres} = \rec n\underbrace{\pd VT\bigg|_\pres}_{=V\beta} \underbrace{\z{\pres + \pd UV\bigg|_T}}_{=\frac{T\beta}{\kappa}} = \frac{T V \beta^2}{n\kappa}.
\end{align}
A zárójeles összefüggést most még nem tudjuk belátni, majd a (\ref{eq:RM_Maxwell}). egyenletben kapunk rá választ. Ellenőrizzük le a Robert Mayer-egyenletet az ideális gáz paramétereivel\footnote{\,Érdekességképp megjegyezzük, hogy van der Waals-gáz esetén a Robert Mayer-egyenlet eredménye: $$C_\pres-C_V = \frac{R}{1-\dfrac{2an}{RT}\dfrac{(V-bn)^2}{V^3}}.$$}:
\begin{align}
    V = \frac{nRT}{\pres},\quad \beta = \rec T,\quad \kappa = \rec \pres \follows C_\pres-C_V = \frac{T V \beta^2}{n\kappa} \stackrel{\tn{id.g.}}= R. 
\end{align}
Eddig a fajhőre illetve mólhőre koncentráltunk az izochor illetve izobár folyamatok esetén, azonban a következőkben részletesebben elemezzük még a nyílt folyamatokat.

\emph{Adiabatikus folyamatnak} nevezzük azt a kvázisztatikus folyamatot, amikor nincs hőcsere, azaz $\delta Q =0$. Ekkor az I. főtétel egyszerűbb alakot ölt:
\begin{align}
    \m dU = -\pres\m dV + \underbrace{\delta Q}_{=0} = -\pres \m dV. 
\end{align}
Fejezzük ki a belső energia teljes deriváltját:
\begin{align}
    \m dU(\pres,V) = \pd U\pres\bigg|_V\m d\pres + \pd UV\bigg|_\pres \m dV = -\pres\m dV\follows \pd U\pres\bigg|_V \m d\pres = -\z{\pres+\pd UV\bigg|_\pres}\m dV,
\end{align}
ami egy
\begin{align}
    \md \pres V = \frac{g(\pres,V)}{f(\pres,V)}
\end{align}
alakú differenciálegyenlet, $f$ és $g$ pedig analitikus függvényei a nyomásnak és a térfogatnak. Nézzük meg ennek a speciális esetét \emph{ideális gázra}. Ekkor a belső energia:
\begin{align}
    U(\pres,V) = \frac f2 \pres V \follows \frac f2 V\m d\pres = -\z{\pres+\frac f2 \pres}\m dV = -\frac{f+2}{2}\pres\m dV,
\end{align}
így a megoldandó differenciálegyenlet a következő:
\begin{align}
    \md \pres V = -\frac{f+2}{f}\frac \pres V \follows \frac{\m d\pres}{\pres} = -\frac{f+2}{f}\frac{\m dV}{V} \follows \m d(\ln \pres) = -\frac{f+2}{f}\m d(\ln V).
\end{align}
Véve mindkét oldal határozatlan integrálját:
\begin{align}\label{eq:adiabata}
    \int \m d(\ln \pres) = \int -\frac{f+2}{f}\m d(\ln V) \follows \ln \pres = \ln\big(V^{-\frac{f+2}{f}}\big)+\tilde C,
\end{align}
ahol $\tilde C$ egy integrálási konstans. Vezessük most be $\kappa$-t, mint:
\begin{align}
    \kappa = \frac{C_\pres}{C_V} = \frac{f+2}{f},
\end{align}
amit adiabatikus kitevőnek nevezünk. Felhívjuk a figyelmet arra, hogy ez nem ugyanaz a $\kappa$, amit a kompresszibilitásra használtunk (ez egy dimenziótlan szám). A félév hátralévő részében $\kappa$ alatt mindig az adiabatikus kitevőt értjük.

A (\ref{eq:adiabata}). egyenletet így a következő alakra hozhatjuk:
\begin{align}
    \pres = \tilde C V^{-\kappa}\follows \pres V^\kappa = \tilde C  = \tn{konst.}
\end{align}

Vizsgáljuk meg azt az általános \emph{kvázisztatikus} folyamatot, melyben a mólhő konstans; az ilyet \emph{politróp folyamatnak} nevezzük. Látni fogjuk, hogy ez tartalmazza az egyszerűbb nyílt folyamatokat speciális esetként. A hőközlés a mólhővel kifejezve:
\begin{align}
    \delta Q = nC\m dT \follows \m dU = -\pres \m dV + \delta Q = -\pres \m dV +nC\m dT,
\end{align}
továbbá a belső energia felírható, mint:
\begin{align}
    \m dU = \pd UV\bigg|_T \m dV + \pd UT\bigg|_V \m dT = -\pres \m dV +nC\m dT.
\end{align}
Most vizsgáljuk meg az \emph{ideális gáz} speciális esetét. Ekkor a belső energia a térfogattal és a hőmérséklettel kifejezve\footnote{\,Emlékezzünk vissza a (\ref{eq:izochor_U}). és (\ref{eq:cv_cp}). egyenletekre, hogy ilyenkor az állandó hőmérsékletű mólhő jelenik meg.}:
\begin{align}
    \m dU(V,T) = nC_V\m dT = -\pres\m dV + nC\m dT
\end{align}
Felhasználva az ideális gáz állapotegyenletét az előbbi egyenlet átrendezhető:
\begin{align}
    \frac{nRT}{V}\m dV = n(C-C_V)\m dT \Rightarrow \frac{\m dV}{V}=\frac{C-C_V}{R}\frac{\m dT}{T}\Rightarrow \m d(\ln V) = \frac{C-C_V}{R}\m d(\ln T).
\end{align}
Az utóbbi egyenlet mindkét oldalát integrálva kapjuk:
\begin{align}
    \int_{V_0}^V\m d(\ln V) = \int_{T_0}^T\frac{C-C_V}{R}\m d(\ln T)\follows \ln\z{\frac{V}{V_0}} = \frac{C-C_V}{R}\ln\z{\frac{T}{T_0}}, 
\end{align}
amiből némi algebrai átalakítással kapjuk, hogy:
\begin{align}
    \frac{V}{V_0} = \z{\frac{T}{T_0}}^{\frac{C-C_V}{R}}=\z{\frac{T_0}{T}}^{\frac{C_V-C}{R}}\follows VT_{\textcolor{white}{0}}^{\frac{C_V-C}{R}}=V_0T_0^{\frac{C_V-C}{R}},\quad TV_{\textcolor{white}{0}}^{\frac{R}{C_V-C}}=T_0V_0^{\frac{R}{C_V-C}}
\end{align}
Ha pedig felhasználjuk az ideális gáz állapotegyenletét, akkor a következőre jutunk:
\begin{align}
    \frac{\pres V}{nR}V_{\textcolor{white}{0}}^{\frac{R}{C_V-C}}=\frac{\pres_0 V_0}{nR}V_0^{\frac{R}{C_V-C}}\follows \pres V^k =\pres_0V_0^k=\tn{konst.},
\end{align}
ahol bevezettük a politropikus kitevőt:
\begin{align}
    k:=\frac{C_\pres-C}{C_V-C}.
\end{align}
Ebből látható, hogy ha $C=0$, akkor $k=\frac{C_\pres}{C_V}=\kappa$, ami visszaadja az adiabatikus kitevőt.

Politróp folyamatnál ideális gáz esetén ahogy már korábban is használtuk a belső energia $\Delta U=n C_V(\underbrace{T_2-T_1}_{\Delta T})$, amiből kiszámolható, hogy a munkavégzés:
\begin{align}
    W = - \frac{\pres_1V_1}{k-1}\sz{1-\z{\frac{V_1}{V_2}}^{k-1}},\quad\tn{illetve a hő}\quad Q=nC(T_2-T_1).
\end{align}

\emph{Izoterm folyamatról} beszélünk, ha a folyamat során a hőmérséklet állandó. Az ideális gázra érvényes Boyle--Mariotte-törvény értelmében ekkor $\pres V$ is állandó, azaz $k = 1 = \frac{C_\pres-\infty}{C_V-\infty}$, azaz $C=\infty$ a politróp modellből. Mivel a belső energia megváltozása arányos a hőmérsékletváltozással, ezért $\Delta U = 0$ a folyamat során. A munkavégzést meghatározhatjuk, mint:
\begin{align}
    W = -\int_{V_1}^{V_2}p\m dV = -\int_{V_1}^{V_2} \frac{nRT}{V}\m dV = -nRT\ln V\Big|_{V_1}^{V_2}= -nRT\ln\z{\frac{V_2}{V_1}}.
\end{align}
A termodinamika I. főtétele alapján:
\begin{align}
    0 = \Delta U = W+Q,\quad\tn{ha } W = -nRT\ln\z{\frac{V_2}{V_1}}\follows Q = nRT\ln\z{\frac{V_2}{V_1}}.
\end{align}
A már korábban részletezett \emph{izochor} folyamatnál a térfogat állandó, a mólhő pedig $C=C_V$, azaz a politropikus együttható $k=\infty$. Ebből látható, hogy ha:
\begin{align}
    \pres V^k = \tn{konst.}\follows \pres^{1/k}V = \tn{konst'.},\quad \tn{tehát } k\to\infty\tn{ esetén valóban } V=\tn{állandó}.
\end{align}
Mivel izochor folyamatoknál $W=0$, ezért:
\begin{align}
    \Delta U = nC_V(T_2-T_1),\quad Q = nC_V(T_2-T_1).
\end{align}
Az állandó nyomású nyílt folyamatot \emph{izobár folyamatnak} nevezzük. Ekkor $C=C_\pres$, tehát $k=0$, így könnyen látszik, hogy valóban ha:
\begin{align}
    k=0 \follows \pres V^0=\tn{konst.} \follows \pres = \tn{állandó}.
\end{align}
Az eddigiekből egyszerűen adódik a belső energia megváltozására, a külső munkavégzésre és a felvett hőre, hogy:
\begin{align}
    \Delta U = nC_V(T_2-T_1),\quad W = -\pres(V_2-V_1),\quad Q = nC_\pres (T_2-T_1).
\end{align}

Az eddigi megállapításainkat \aref{tab:nyilt_foly}. táblázatban összegezzük.
\begin{table}[htb]
\centering
\begin{tabular}{|c|c|c|c|} \hline
Folyamat & $\Delta U_{12}$ & $W_{12}$ & $Q_{12}$\\ \hline\hline
izoterm ($T=\tn{áll.}$) & 0 & $-nRT\ln\z{\dfrac{V_2}{V_1}}$ & $nRT\ln\z{\dfrac{V_2}{V_1}}$\\ \hline
izobár ($\pres=\tn{áll.}$) & $nC_V(T_2-T_1)$ & $-\pres(V_2-V_1)$ & $nC_\pres(T_2-T_1)$ \\ \hline
izochor ($V=\tn{áll.}$) & $nC_V(T_2-T_1)$ & 0 & $nC_V(T_2-T_1)$\\ \hline
adiabatikus ($Q=0$) & $nC_V(T_2-T_1)$ & $- \dfrac{\pres_1V_1}{\kappa-1}\z{1-\z{\dfrac{V_1}{V_2}}^{\kappa-1}}$ & 0\\ \hline
politróp ($C=\tn{áll.}$) & $nC_V(T_2-T_1)$ & $- \dfrac{\pres_1V_1}{k-1}\z{1-\z{\dfrac{V_1}{V_2}}^{k-1}}$ & $nC(T_2-T_1)$\\ \hline
\end{tabular}
\caption{Egyszerű nyílt folyamatok belső energiájának, munkavégzésének és felvett hőjének kiszámolása két pont között.}
\label{tab:nyilt_foly}
\end{table}
