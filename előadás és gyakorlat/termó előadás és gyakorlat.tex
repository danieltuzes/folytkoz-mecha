\documentclass[12pt,a4paper]{scrartcl}

\usepackage{graphicx}
\usepackage{amsmath}
\usepackage{amsfonts}
\usepackage{amssymb}
\usepackage{bm}
\let\mathbf\bm


\usepackage[T1]{fontenc}
\usepackage[utf8]{inputenc}
\usepackage[magyar]{babel}
\usepackage{lmodern}

\usepackage{placeins}	%FloatBarrier parancs, a figure előtti szöveget nem ír ki, amíg a képet sikeresen be nem ágyazza
\usepackage{subcaption}	%alábra lehetősége
\usepackage{epstopdf}
\usepackage{xcolor}
\usepackage[hidelinks,unicode]{hyperref}
\hypersetup{
    colorlinks,
    linkcolor={red!50!black},
    citecolor={blue!50!black},
    urlcolor={blue!80!black}
}

\usepackage{cleveref}	




\begin{document}
\title{Hőtan és folytonos közegek mechanikája\\(emelt szint)}
\author{hotanef19va, Anyagfizikai Tanszék\\
Előadás: Ispánovity Péter Dusán\\
Gyakorlat: Tüzes Dániel}
\maketitle
\tableofcontents
\section*{Mi ez?}
A 2019/2020-as tanévtől kezdődően a hőtan és folytonos közegek mechanikája egy tárgyban kerül meghirdetésre. A téma a szemeszter első felében elsősorban a folytonos közegek mechanikája, második felében a hőtan. Ez a jegyzet az előadáshoz és a gyakorlathoz közösen írt tananyag, amely a hőtan részt fedezi. További támogató anyagok esetleg elérhetőek az ELTE Canvas rendszerében (mint pl.\ előadás videók), az
\href{http://ispanovity.web.elte.hu/teaching/}{előadó} vagy a  \href{http://metal.elte.hu/~tuzes/oktatas/#folytkoz_19_20}{gyakorlatvezető} honlapján (mint pl.\ házi feladatok).

Külön köszönet illeti meg Boldizsár Bálint hallgatót, aki nemcsak az előadáshoz tartozó demonstrációban, de ennek a jegyzetnek az elkészítésében is szerepet vállalt.

\section{A termodinamika alapfogalmai}
~
\section{Anyagi tulajdonságok}
~
\section{Kinetikus gázelmélet}
~
\section{Reális gázok}
~
\section{A termodinamika I. főtétele}
~
\section{Folyamatok és hőtani alapmennyiségek}
~
\section{I.\ főtétel alkalmazásai}
~
\section{A termodinamika II.\ főtétele}
~
\section{Carnot-féle körfolyamat}
~
\section{Entrópia}
~
\section{Ideális gáz entrópiája}
~
\section{Termodinamikai potenciálok}
~
\section{II.\ főtétel következményei}
~
\section{Összetett rendszerek egyensúlya}
~
\section{Oldhatóság}
~
\section{Kanonikus sokaság}
~
\section{Egykomponensű rendszer egyensúlya}
~
\section{Fázisátalakulások}
~
\section{Hang terjedése gázban}
~
\section{Rugalmas hullámok terjedése}
~
\end{document}

