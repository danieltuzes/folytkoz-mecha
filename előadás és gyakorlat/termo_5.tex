\section{A termodinamika I.~főtétele}
\emph{Hő fogalma és kialakulása; Munka és hő ekvivalenciája; Belső energia; Joule-kísérlet; I. főtétel; Energiamegmaradás elve; Kvázisztatikus ill. adiabatikus folyamatok; Ideális gáz és vdW-gáz belső energiája.}

Sokszor használtuk már eddig is a \emph{hő} fogalmát, azonban nem olyan egyszerű megfogalmazni, hogy mi is az pontosan. Régen még csak az sem volt egyértelmű, hatalmas viták voltak róla, hogy ez anyag (hőmennyiség) vagy inkább mozgás-e (munkavégzés).

Ennek eldöntésére Joule\footnote{\,James Prescott Joule, 1818-1889.} végzett el kísérletet, azonban előbb tekintsük át a történeti előzményeket. 1738-ban ahogy már részleteztük, Bernoulli publikálta a kinetikus gázelméletet. 1763-ban Watt\footnote{\,James Watt, 1736-1819.} elkészítette az első gőzgépet, ezzel megindult a hő gyakorlati felhasználása anélkül, hogy tudták is volna valójában, hogy mi az. 1783-ban Lavoisier\footnote{\,Antoine Lavoisier, 1743-1794.} és Black\footnote{\,Joseph Black, 1728-1799.} felállították a hőanyag elméletüket. Az 1790-es években Thompson\footnote{\,Benjamin Thompson, 1753-1814.} előállt azzal a gondolattal, miszerint ,,a hő mozgás'', azaz munkavégzéssel is lehet hőt termelni. 1824-ben használta először Clément\footnote{\,Nicolas Clément, 1779-1841.} a kalória fogalmát (1 cal az a hőanyag, ami 1g víz 1$^\circ$C-kal való felmelegítéséhez szükséges, $1\tn{ cal} = 4{,}184$J). Végezetül következett Joule kísérlete 1842-ben: hőszigetelt, a környezettől elzárt, vízzel teli tartályban lapátokkal ellátott függőleges rudat hajtott meg egy súly segítségével (a súly rögzítése egy csigán volt átvetve, a vége pedig a rúdra ráerősítve, így a gravitációs tér a súlyt a föld felé vonzotta, a rúd pedig forgásba jött). A súly helyzeti energiája a lapátok mozgási energiájává alakult, ami a vízzel való súrlódáson keresztül felmelegítette azt. A \emph{hő} tehát az \emph{energiaáramlás során átadott energia} a kísérlet alapján. Munkát végezhetünk például súrlódással, közegellenállással vagy összenyomással is.

A kísérletben ha a tartály \emph{hőszigelelt}, bárhogy is végezzük el a kísérletet, de a súlyok kezdő- és végpontja ugyanaz, akkor a munkavégzés is megegyezik, azaz a $W_{AB}$ $A$ és $B$ pontok közötti munkavégzés nem függ az úttól. Jelölje $U$ a rendszer belső energiáját. Ekkor a $B$ pontban lévő súly esetén a rendszer belső energiája egyértelműen kifejezhető az $A$ pontban lévő súly melletti belső energiával mint:
\begin{align}
    U_B = U_A + W_{AB},\quad \tn{vagy más jelöléssel}\quad U(B) = U(A) + W_{AB}.
\end{align}
Azaz ha egy helyen lerögzítjük a belső energiát, akkor az egyértelműen meghatározza az értékét bármely más helyen is. Az $U$ \emph{állapotjelző}.

Két fizikailag alapvetően különböző módon lehet munkát végezni. Ennek szemléltetéséhez képzeljünk el egy dugattyút.
\begin{itemize}
    \item A dugattyú tömege legyen $m$, továbbá legyen még egy $M$ tömegű súlyunk is. Ebben az esetben a súlyt hirtelen rárakjuk a dugattyúra, majd amikor az összenyomódott, levesszük a súlyt. Ekkor a teljes munkavégzés a folyamat végéig $$W=(M+m)gh-mgh = Mgh,$$ azaz a végén a dugattyú teteje magasabban lesz, megnő a bezárt gáz térfogata. Az ilyen folyamatokat \emph{irreverzibilis} folyamatnak nevezzük. Az összes kiegyenlítődési folyamat irreverzibilis. Az eddig vizsgált intenzív állapotjelzőink egyensúlyban értelmezett mennyiségek, így például nincs értelme a súly rárakása és a levétele közben arról beszélni, hogy mekkora a nyomás a dugattyúban (hiszen közvetlenül a dugattyú alatt nagyobb a nyomás, mint a tartály alján). Ennek megfelelően ha a folyamatot $\pres-V$ diagramon akarnánk ábrázolni, azt nem köthetjük össze folytonos vonallal (ráadásul itt nem is izotermán mozogna a rendszer, mert közben felmelegszik).
    \item Másik esetben ugyanúgy $m$ tömegű a dugattyú, azonban a $M$ tömeget most porszemcsék formájában folyamatosan pakoljuk rá, majd vesszük le a végén. Ekkor minden porszemcse rárakása illetve levétele után a rendszer egyensúlyban van. Az ilyen folyamatokat \emph{kvázisztatikusnak} nevezzük. Ha minden állapotban a rendszer egyensúlyban van, akkor az állapotjelzők is értelmezve vannak, így a $\pres-V$ diagramon összeköthetjük folytonos vonallal az izotermát.
\end{itemize}
A továbbiakban ha másképp nem jelezzük mindig kvázisztatikus folyamatokkal foglalkozunk. Vizsgáljunk egy $A$ keresztmetszetű hengeres dugattyút. Tudjuk, hogy adott nagyságú $F$ erővel a $\Delta x$ kis elmozduláshoz tartozó munkavégzés (a dugattyú alakja miatt $\bm F \parallel \Delta \bm x$):
\begin{align}
    \Delta W = F\Delta x = \frac FA A\Delta x = -\pres\Delta V.
\end{align}
A negatív előjel azért jelent meg, mert megállapodás szerint a munka akkor pozitív, ha a külső erő végzi a rendszeren a munkát, azaz a térfogat csökken. $\pres-V$ diagram $A$ és $B$ pontja között végzett teljes munka ekkor:
\begin{align}
    W = -\int_A^B \pres\m dV,
\end{align}
azonban fontos, hogy ez csak \emph{kvázisztatikus} folyamatokra igaz, hiszen a nyomásnak értelmezettnek kell lennie a teljes folyamat során. Ebből láthatjuk, hogy ha munkavégzéssel eljutunk $A$-ból $B$-be, akkor nem lehet ugyanúgy külső munkavégzéssel $B$-ből $A$-ba is eljutni (kivéve a később tárgyalt adiabatikus folyamatoknál). Ennek megfelelően létezik egy olyan $U$ mennyiség - a \emph{belső energia} - amire igaz, hogy:
\begin{align}\label{eq:U_int}
    0 = \oint \m dU = \oint\z{\pd{U}{\pres}\bigg|_V\m d\pres + \pd{U}{V}\bigg|_\pres \m dV} = \oint
    \underbrace{\begin{pmatrix}
    \dfrac{\partial U}{\partial \pres}\bigg|_V & \dfrac{\partial U}{\partial V}\bigg|_\pres
    \end{pmatrix}}_{\equiv\bm\nabla U}
    \begin{pmatrix}
    \m d\pres \\ \m dV
    \end{pmatrix} = \oint \bm\nabla U \m d\bm r,
\end{align}
ahol $\bm r=(\pres,V)$ és $\m d\bm r = (\m d\pres, \m dV)$.
Ha nincs a rendszer hőszigetelve, akkor ha $U(B)~{-}~U(A)$ mennyiség állandó, viszont a $\pres-v$ digramon más úton jutunk el $A$-ból $B$-be, akkor $W_1\neq W_2$, ahol az indexek az utak különbözőségére utalnak. Ebből következik, hogy:
\begin{align}
    U(B)-U(A) = W_1+Q_1 = W_2+Q_2,
\end{align}
ahol $Q_1$ és $Q_2$ az adott úthoz tartozó hőt jelöli. Az irodalomban szokásos jelölésekkel ezt úgy foglalhatjuk össze, hogy:
\begin{align}
    \m dU = \delta W + \delta Q,
\end{align}
ez a \emph{hőtan (termodinamika) I. főtétele}. A hő ebből definíció szerint $\delta Q = \m dU-\delta W$. A belső energia tehát akkor változik meg, ha a gázon munkát végzünk vagy vele hőt közlünk. Továbbá felhasználva a (\ref{eq:U_int}). összefüggést:
\begin{align}
    0 = \oint \m dU = \underbrace{\oint \delta W}_{\neq 0} + \oint \delta Q,
\end{align}
így láthatjuk, hogy $W$ és $Q$ egyike sem állapotjelző\footnote{\,Ha az $A$ mennyiség állapotjelző, akkor rá teljesül, hogy $$\oint \m dA=0.$$}.
\emph{Adiabatikus folyamatnak} nevezzük az olyan folyamatokat, amikor nem történik hőcsere, továbbá a folyamat kvázisztatikus. Ha a folyamat nem kvázisztatikus, akkor biztosan nem adiabatán fog mozogni. Minden termodinamikai folyamatra igaz, hogy:
\begin{align}
    \Delta U = W +Q,
\end{align}
azonban csak kvázisztatikus folyamat esetén igaz\footnote{\,Érdekességképp megjegyezzük, hogy ha lenne mondjuk külső mágneses tér, akkor annak a járulékát is be kellene írni a munkavégzéssel, de ilyen esetekkel a kurzus során nem foglalkozunk.}, hogy:
\begin{align}
    \m dU = \delta W+ \delta Q = -\pres\m dV +\delta Q.
\end{align}
A hőtan I. főtétele megfogalmazható másképp is, miszerint nem létezik \emph{elsőfajú perpetuum mobile}, azaz nincs olyan gép, ami önmagától energiát termel (tehát a részecskék közötti kölcsönhatás konzervatív).

Az ideális gáz kinetikus gázmodellje kapcsán meghatároztuk már korábban a gázrészecskék kinetikus energiáját, azaz a gáz belső energiáját, ami:
\begin{align}
    U_{\tn{id.g}}(V,T) = \frac f2 nRT,
\end{align}
ha pedig a belső energia más változókkal adott, akkor az állapotegyenletet felhasználva:
\begin{align}
    U_{\tn{id.g}}(V,T) = \frac f2 nRT,\quad U_{\tn{id.g}}(\pres,T) = \frac f2 nRT,\quad  U_{\tn{id.g}}(\pres,V) = \frac f2 \pres V.
\end{align}
%\newgeometry{includefoot,left=2cm,right=2cm,bottom=1cm,top=2cm}
Belátható\footnote{\,Tekintsünk egy zárt hengert, ami van der Waals-típusú reális gázzal van kitöltve. A dugattyút mozgatva a gázon végzett munka ekkor tisztán a belső energia megváltozására fordítódik. A részecskék molekuláris kölcsönhatásokból eredő belső energia járuléka ekkor ha a $bn$ minimális térfogatról $V$ térfogatra növeljük :$$U_{\tn{kh.}} = \int_{bn}^V p_{\tn{k.h.}}\m dV' = \int_{bn}^V \frac{an^2}{{V'}^2}\m dV' = -\frac{an^2}{V}+\frac{an}{b}.$$
A gáz teljes energiája ekkor a kinetikus illetve potenciális tagokkal együtt:
$$U = U_{\tn{id.g.}} + U_{\tn{k.h.}} = \frac f2 nRT -\frac{an^2}{V}+\frac{an}{b}.$$
Felhasználjuk azonban még, hogy a belső energia nullpontját a molekulák közötti kölcsönhatás rövid hatótávolságát kuhasználva úgy rögzítjük, hogy a $V\to\infty$ határesetben a van der Waals-gáz belső energiája az ideális gázéba menjen át, így végezetül a van der Waals-gáz belső energiája:
\begin{align}
    U_{\tn{vdW}}(V,T) = \frac f2 nRT - a\frac{n^2}{V}.
\end{align}}, hogy a van der Waals-gáz belső energiája:
\begin{align}
    U_{\tn{vdW}}(V,T) = \frac f2 nRT - a\frac{n^2}{V}.
\end{align}
%\restoregeometry