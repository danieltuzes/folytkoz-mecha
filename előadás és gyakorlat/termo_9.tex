\section{Carnot-féle körfolyamat}
\emph{Carnot-körfolyamat fogalma; Hatásfok ideális gáz esetén; Hűtőgép és hőszivattyú hatásfoka; Termodinamikai hőmérsékleti skála; Carnot-körfolyamat speciális tulajdonságai; Kelvin--Planck-gép és Clausius-gép ekvivalenciája; Stirling-motor és hatásfoka}

Emlékezzünk vissza, hogy körfolyamatoknak nevezzük azokat a termodinamikai folyamatokat, melyek során az állapotjelzők a kiindulási értékeiket újra felveszik, úgy, hogy a folyamat közben nem zérus hőcsere és munkavégzés történik. Próbáljunk konstruálni egy periodikus hőerőgépet, méghozzá a lehető legjobb hatásfokút! \emph{Carnot\footnote{\,Nicolas Léonard Sadi Carnot, 1796-1832.}-körfolyamatnak} (1824.) nevezzük az olyan körfolyamatokat, melyek két adiabatát és két izotermát tartalmaznak. A $T_1$ és $T_2$ hőmérsékletű hőtartályokkal van hőcsere, az adiabatán értelemszerűen nincs. \Aref{tab:Carnot}. táblázatban foglaljuk össze az egyes szakaszokon történő belső energia változást, munkavégzést és hőcserét.
\begin{table}[h!]
\centering
\begin{tabular}{|c|c|c|c|} \cline{2-4}
\multicolumn{1}{c|}{} & $\Delta U$ & $W$ & $Q$\\ \hline\hline
$1\to 2$ & 0 & $-nRT_1\ln\z{\frac{V_2}{V_1}}$ & $nRT_1\ln\z{\frac{V_2}{V_1}}$ \\ \hline
$2\to 3$ & $-nC_V(T_1-T_2)$ & $-nC_V(T_1-T_2)$ & 0 \\ \hline
$3\to 4$ & 0 & $nRT_2\ln\z{\frac{V_3}{V_4}}$ & $-nRT_2\ln\z{\frac{V_3}{V_4}}$\\ \hline
$4\to 1$ & $nC_V(T_1-T_2)$ & $nC_V(T_1-T_2)$ & 0\\ \hline
\end{tabular}
\caption{Carnot-folyamat szakaszain történő belső energia megváltozása, a munkavégzés mértéke, illetve a hőcsere.}
\label{tab:Carnot}
\end{table}
A teljes munkavégzés az egyes szakaszokon történő munkavégzések össze, mely a táblázat alapján:
\begin{align}
	W = W_{12}+W_{23}+W_{34}+W_{41} = nR\kz{T_2\ln\z{\frac{V_3}{V_4}}-T_1\ln\z{\frac{V_2}{V_1}}}.
\end{align}
Láttuk már a nyílt folyamatok tárgyalásakor, hogy ideális gáz esetén adiabatikus folyamatoknál:
\begin{align}
	pV^\kappa = \tn{konst.},\quad\tn{illetve}\quad TV^{\kappa-1}=\tn{konst'.}
\end{align} 
Ezt felhasználva az adiabatikus szakaszokra adódik, hogy:
\begin{align}\label{eq:Carnot_ad}
	T_1V_1^{\kappa-1}=T_2V_4^{\kappa-1},\quad T_1V_2^{\kappa-1}=T_2V_3^{\kappa-1} \Rightarrow \z{\frac{V_1}{V_2}}^{\kappa-1} = \z{\frac{V_4}{V_3}}^{\kappa-1} \Rightarrow \frac{V_1}{V_2}=\frac{V_4}{V_3}.
\end{align}
A teljes folyamatra:
\begin{align}
	\underbrace{W_{12}+W_{23}+W_{34}+W_{41}}_{=W}+Q_{12}+\underbrace{Q_{23}}_{=0}+Q_{34}+\underbrace{Q_{41}}_{=0} = 0,
\end{align}
amit tovább írhatunk, mint:
\begin{align}
	W=-Q_{12}-Q_{34} = -Q_{12}+|Q_{34}| = -Q_{12}\z{1-\frac{|Q_{34}|}{Q_{12}}} = -Q_{12}\z{1-\frac{T_2}{T_1}},
\end{align}
ahol felhasználtuk, hogy \aref{tab:Carnot}. táblázat és a (\ref{eq:Carnot_ad}). összefüggés alapján
\begin{align}
	\frac{Q_{12}}{|Q_{34}|} = \frac{T_1}{T_2}.
\end{align}
Definiáljuk a \emph{hatásfokot}, mint:
\begin{align}
	\eta := \frac{|W|}{Q_{\tn{fel}}} \stackrel{\tn{most}}{=} \frac{-W}{Q_{12}},
\end{align}
ami \emph{Carnot-folyamat} esetén az alábbi alakot ölti:
\begin{align}
	\eta_{\tn{id.g.}}^{\tn{Carnot}}= \frac{T_1-T_2}{T_1} = 1-\frac{T_2}{T_1}<1,
\end{align}
azaz nem lehet a hatásfok 1-nél nagyobb.
Véssük nagyon az eszünkbe, hogy míg a gáz által végzett munkával és a felvett hővel a hatásfok kifejezése általánosan is, azonban az itt a hőtartályok hőmérsékletével megadott kifejezés csak \emph{Carnot-folyamat} esetén igaz.
Mivel Carnot-folyamatnál:
\begin{align}\label{eq:Carnot_S}
	1-\frac{T_2}{T_1} = 1-\frac{|Q_{34}|}{Q_{12}} = 1+\frac{Q_{34}}{Q_{12}}\follows \frac{Q_{12}}{T_1}+\frac{Q_{34}}{T_2} = 0,
\end{align}
azaz láthatjuk, hogy a Carnot-folyamatra az alábbi mennyiség körintegrálja eltűnik:
\begin{align}
	\oint \frac{\delta Q}{T} = 0,
\end{align}
amivel a későbbiekben még részletesebben is fogunk foglalkozni.
Láthatjuk a hatásfoknál, hogy minél nagyobb $T_1$ a $T_2$-höz képest, annál jobb lesz a hatásfok\footnote{\,Emiatt például az erőműveknél is olyan anyagokat keresnek, amik magasabb hőmérsékletet is kibírnak.}.
A Carnot-folyamat megfordítható, ráadásul ez az \emph{egyetlen két hőtartályú reverzibilis} körfolyamat. A megfordított Carnot-folyamatot hűtőgépnek, illetve hőszivattyúnak\footnote{\,Mindkettő ugyanazt jelöli, hűtőgépnek nevezzük, amikor az a fontos, hogy hideg helyről von el hőt, hőszivattyúként hivatkozunk pedig rá, ha az a fontos, hogy fűtésre használjuk.} nevezzük. 
A hőszivattyú hatásfokát \emph{jósági tényezőnek} nevezzük, melyre igaz, hogy:
\begin{align}
	\eta_{\tn{hőszivattyú}}^{\tn{Carnot}} = \frac{|Q_{12}|}{-W} = \frac{T_1}{T_1-T_2} = \frac{1}{\eta_{\tn{id.g.}}^{\tn{Carnot}}}>1
\end{align}
Ezzel szemben a hűtőgép hatásfoka:
\begin{align}
	\eta_{\tn{hűtőgép}}^{\tn{Carnot}} = \frac{|Q_{34}|}{W} = \frac{T_2}{T_1}\cdot Q_{12}\cdot \frac{T_1}{T_1-T_2} = \frac{T_2}{T_1-T_2} = \eta_{\tn{hőszivattyú}}^{\tn{Carnot}}-1>0.
\end{align}
Vizsgáljuk meg azt a korábbi állításunkat, miszerint a termodinamika II. főtételének Clausius- illetve Kelvin--Planck-féle megfogalmazása egymással ekvivalens. Emlékeztetőként Clausius-féle gépnek nevezzük azt a gépet, aminél hő áramolna alacsonyabb hőmérsékletű helyről magasabb hőmérsékletű helyre a környezeten végzett munka nélkül. Kelvin--Planck-gépnek hívjuk az olyan periodikusan működő gépet, mely csak egy hőtartállyal áll kapcsolatban, mégis munkát végez. A II. főtétel értelmében ezek egymással \emph{ekvivalensek} és \emph{egyik sem létezik}.

Tegyük fel, hogy van egy Clausius-gépünk, amihez Carnot-folyamatot kapcsolunk. Ekkor ha a Clausius gépben áramlő hő $Q$, a Carnot-folyamatba bemenő hő szintén $Q$, az abból kiáramló hő pedig $Q'$, akkor:
\begin{align}
	Q' = \frac{T_2}{T_1}Q<Q,\quad\tn{továbbá}\quad Q-Q'=W,
\end{align}	
azaz eredményül pont egy Kelvin--Planck-gépet kaptunk, hiszen kivesz a $T_2$ hőmérsékletű hőtartályból $Q-Q'$ hőt, és ezzel $W$ munkát végez. Nézzük most a fordított esetet, azaz tegyük fel, hogy létezik Kelvin--Planck-gép. Ekkor ha hozzákapcsolunk egy visszafelé járatott Carnot-folyamatot, hasonló gondolatmenettel Clausius-géphez jutnánk, azonban a II. főtétel értelmében egyik sem létezhet. 

Összegezzük most az eddigi tapasztalatainkat a \emph{Carnot-folyamatokról}:
\begin{itemize}
	\item periodikus, reverzibilis gép a lehető legkevesebb hőtartállyal\footnote{\,Emlékezzünk rá, hogy az egy hőtartályos gép pont a Kelvin--Planck-gép lenne.} és legnagyobb hatásfokkal (utóbbira még később visszatérünk),
	\item a folyamat megfordítható,
	\item ha egy körfolyamat bármely szakaszán irreverzibilitás lép fel, akkor kisebb lesz a hatásfoka, mint a reverzibilis Carnot-folyamatnak, azaz $\eta_{\tn{irrev.}} < \eta_{\tn{rev.}}$
	\item a Carnot-körfolyamat hatásfoka nem függ a benne lévő anyagtól.
\end{itemize}
Az utóbbi állítást még meg kell vizsgálnunk részletesebben. Észrevehettük már korábban, hogy a Carnot-folyamat hatásfokánál mindig automatikusan felhasználtuk, hogy az:
\begin{align}
	\eta^{\tn{Carnot}}=1-\frac{T_2}{T_1},
\end{align}
pedig a levezetést ideális gázra végeztük el. Tegyük fel, hogy valamilyen más anyag esetén a Carnot-körfolyamat hatásfoka nagyobb, mint ideális gáz esetén, azaz:
\begin{align}
	\eta_{\tn{akármi}}^{\tn{Carnot}} > \eta_{\tn{id.g.}}^{\tn{Carnot}}.
\end{align}
A munkavégzés mindkét esetben megegyezik, tehát:
\begin{align}
	-W = \eta_{\tn{akármi}}^{\tn{Carnot}}\cdot Q_{12}' = \eta_{\tn{id.g.}}^{\tn{Carnot}}\cdot Q_{12} \follows \frac{Q_{12}}{Q_{12}'}=\frac{\eta_{\tn{akármi}}^{\tn{Carnot}}}{\eta_{\tn{id.g.}}^{\tn{Carnot}}}>1,
\end{align}
ami így egy Clausius-gép lenne, ellentmondásba ütköztünk. Tegyük fel most az állítás megfordítását, azaz hogy:
\begin{align}
	\eta_{\tn{akármi}}^{\tn{Carnot}} < \eta_{\tn{id.g.}}^{\tn{Carnot}},
\end{align}
azonban ebben az esetben is az előbbi megfontolások alapján Clausius-gépet kapnánk (ami most az előzőhöz képest ellentétesen van járatva), tehát ez a feltevésünk is ellentmondásba ütközött, így beláttuk, hogy a \emph{Carnot-folyamat hatásfoka minden anyagra ugyanakkora}\footnote{\,Gyakorlásképp kiszámolhatjuk van der Waals-gázra is, valóban ugyanaz jön ki, mint ideális gáz esetén.}, mégpedig:
\begin{align}
	\eta^{\tn{Carnot}} = 1-\frac{T_2}{T_1}.
\end{align}
A Carnot-folyamat lehetővé teszi számunkra egy \emph{termodinamikai (anyagfüggetlen) hőmérsékletskála} bevezetését. Legyen most $T\equiv T_1$ a mérni kívánt hőmérséklet, $T_{\tn{ref}}\equiv T_2$ pedig a referenciahőmérséklet. Ekkor a mért hőmérséklet kifejezhető, mint:
\begin{align}
	\eta = 1-\frac{T_{\tn{ref}}}{T} \follows T = \frac{T_{\tn{ref}}}{1-\eta}.
\end{align}
Vizsgáljunk meg még másik körfolyamatokat is. A belső égésű motorokat \emph{Otto-motornak} nevezik, ez két adiabatikus és két izochor folyamatból áll\footnote{\,Gyakorlásképp ez is kiszámolható, az Otto-motor hatásfoka ideális gázra:$$\eta_{\tn{id.g.}}^{\tn{Otto}}=1-\z{\frac{V_2}{V_1}}^{\kappa-1},$$
ahol $V_2<V_1$.}.
A külső égésű motort \emph{Stirling-motornak} nevezzük, ez két izoterm és két izochor szakaszból áll, melyeket ideális gáz esetén \aref{tab:Stirling}. táblázatban részletezzük. 
\begin{table}[h!]
\centering
\begin{tabular}{|c|c|c|c|} \cline{2-4}
\multicolumn{1}{c|}{} & $\Delta U$ & $W$ & $Q$\\ \hline\hline
$1\to 2$ & 0 & $-nRT_1\ln\z{\frac{V_2}{V_1}}$ & $nRT_1\ln\z{\frac{V_2}{V_1}}>0$ \\ \hline
$2\to 3$ & $-nC_V(T_1-T_2)$ & 0 & $-nC_V(T_1-T_2)$<0 \\ \hline
$3\to 4$ & 0 & $nRT_2\ln\z{\frac{V_2}{V_1}}$ & $-nRT_2\ln\z{\frac{V_2}{V_1}}<0$\\ \hline
$4\to 1$ & $nC_V(T_1-T_2)$ & 0 & $nC_V(T_1-T_2)>0$\\ \hline
\end{tabular}
\caption{Stirling-folyamat szakaszain történő belső energia megváltozása, a munkavégzés mértéke, illetve a hőcsere ideális gáz esetén.}
\label{tab:Stirling}
\end{table}
A táblázat értékei alapján a hatásfokra kapjuk:
\begin{align}
	\eta_{\tn{id.g.}}^{\tn{Stirling}} = -\frac{W_{12}+W_{34}}{Q_{12}+Q_{41}} = \frac{R(T_1-T_2)\ln\z{\frac{V_2}{V_1}}}{RT_1\ln\z{\frac{V_2}{V_1}}+C_V(T_1-T_2)} = 1-\frac{\frac{T_2-T_1}{\kappa-1}+T_1\ln\z{\frac{V_2}{V_1}}}{\frac{T_2-T_1}{\kappa-1}+T_2\ln\z{\frac{V_2}{V_1}}}.
\end{align}
Például legyen $T_1 = 600$K, $T_2 = 300$K, $\frac{V_2}{V_1} = 2$, $C_V = \frac 32 R$. Ezekkel az adatokkal:
\begin{align}
	\eta_{\tn{id.g.}}^{\tn{Stirling}} \approx 0{,}24,\quad\tn{míg}\quad \eta^{\tn{Carnot}} = 0{,}5,
\end{align}
azaz $\eta_{\tn{id.g.}}^{\tn{Stirling}}<\eta^{\tn{Carnot}}$ ebben a speciális esetben is, ahogyan annak általánosan is igaznak kell lennie két hőtartályos körfolyamatoknál.