\section{Egykomponensű rendszer egyensúlya}
\emph{Egyensúlyi állapot meghatározása; Mérlegszabály; Több változó esete; Stabilitási feltételek}

Használjuk az entrópiamaximum elvét és vizsgáljuk meg, hogy az egy komponensű rendszernek (egy féle anyagból áll) mikor van egyensúlya. Eddig láttuk, hogy két hőtartály esetében akkor van egyensúly, ha mindkét oldalon $U_0$ a belső energia, $V$ a térfogat és $n$ az anyagmennyiség. Ekkor a rendszer teljes entrópiája $2S(U_0,V,n)$ lesz. Tegyük fel most, hogy az anyagunk nem teljesen homogén, hanem két különböző állapotban van jelen, méghozzá $U_0{-}\Delta U_1$ és $U_0{+}\Delta U_2$ belső energiákkal, azonban legyenek továbbra is a térfogatok és az anyagmennyiségek a két térrészben azonosak. Ekkor a teljes entrópia $S(U_0{-}\Delta U_1,V,n)+S(U_0{+}\Delta U_2,V,n)$, ami lehet több is, mint a homogén anyag esetében, tehát az \emph{inhomogenitás növelheti az összentrópiát}. Ez akkor következhet be, ha az entrópia energiától függő függvénye konvex. Láttuk már az összetett rendszernél, hogy akkor van egyensúly, ha $\frac{\partial^2 S}{\partial U^2}<0$, ha ez nem teljesül, azaz a függvény konvex, akkor \emph{szétesik két alrendszerre}, mert így nő az összentrópiája. Vizsgáljunk most ilyen két alrendszerre széteső anyagokat. Az anyag $x_1$ hányada legyen az egyik $x_2$ hányada pedig a másik alrendszerben. Értelemszerűen $x_1+x_2=1$. Ekkor a teljes belső energia felírható a két alrendszer belső energiáinak összegeként:
\begin{align}
	x_1{\cdot}(U_0{-}\Delta U_1) + x_2{\cdot} (U_0{+}\Delta U_2) = U\follows x_1\Delta U_1 = x_2\Delta U_2 \follows x_2 =x_1\frac{\Delta U_1}{\Delta U_2}, 
\end{align}
ez a feltételünk arra, hogy az összenergia ne változzon meg azáltal, hogy az anyag szétesik két alrendszerre. Felhasználva, hogy $x_1+x_2 = 1$ kapjuk a két alrendszer teljes rendszerbeli hányadára, hogy:
\begin{align}
	-x_1\Delta U_1+x_2\Delta U_2 = 0\Rightarrow (x_2-1)\Delta U_1+x_2\Delta U_2=0\Rightarrow x_2(\Delta U_1+\Delta U_2) = \Delta U_1,
\end{align}
amiből adódik, hogy:
\begin{align}
	x_2 = \frac{\Delta U_1}{\Delta U_1{+}\Delta U_2}\quad\tn{és hasonlóan}\quad x_1 = \frac{\Delta U_2}{\Delta U_1{+}\Delta U_2}.
\end{align}
Hogyan változott meg az összentrópia?
\begin{align}\label{eq:merleg}
\begin{split}
	S^* &= x_1S_1(U_1,V,n)+x_2S_2(U_2,V,n) = x_1S_1+x_2S_2=\\
	&=(1-x_2)S_1+x_2S_2 = S_1+x_2(S_2-S_1) = S_1 + \frac{\Delta U_1}{\Delta U_1{+}\Delta U_2}(S_2-S_1).
\end{split}
\end{align}
Az anyag tehát széteshet két részre, ha úgy nagyobb lesz az entrópiája, a (\ref{eq:merleg}). összefüggést hívjuk \emph{mérlegszabálynak}. Mindig az érintőnek az $U_0$ helyen vett értéke lesz az entrópia. Ha az $S(U)$ görbének van konvex és konkáv tartománya is, akkor létezik két olyan pontja a görbének, ahol a hőmérséklet megegyezik, azaz az érintők meredeksége megegyezik ($\rec T = \pd SU$). Ezeket a különböző tulajdonságú pontokat nevezzük \emph{fázisoknak}, közöttük pedig az entrópiában van egy instabil tartomány. Ezek alapján megfogalmazhatjuk a \emph{stabilitási feltételeket}, azaz hogy mikor lesz homogén a rendszer. Mivel az entrópiafüggvénynek konkávnak kell lennie, ezért:
\begin{align}
	\frac{\partial^2 S}{\partial U^2} = \pd{}{U}\z{\pd SU}=\pd{}{U}\z{\rec T} = -\frac{1}{T^2}\pd TU = -\rec{T^2}\rec{\pd UT}=-\rec{T^2}\rec{nC_V}<0,
\end{align}
amiből következik, hogy:
\begin{align}
	C_V>0,
\end{align}
a \emph{mólhő pozitív} kell legyen a stabilitáshoz, ezt hívják \emph{Le Chatelier\footnote{\,Henry Louis Le Chatelier, 1850-1936.}-elvnek}. Ez gyakorlatilag azt jelenti, hogy ha $C_V<0$ lenne, akkor ha hőmérsékletkülönbség alakul ki, akkor hő kezd el áramolni a nagyobb hőmérsékletű helyre, így a rendszer instabil. Mivel az entrópia nem csak a belső energia függvénye, hanem például a térfogaté is, meg kell fogalmaznunk, hogy a stabilitási feltétel hogyan jelenik meg több változó esetében. Ekkor képezzük az $S(U,V)$ függvény \emph{Hesse-mátrixát}:
\begin{align}
	\bm H =
	\begin{pmatrix}
	\dfrac{\partial^2 S}{\partial U^2} & \dfrac{\partial ^2 S}{\partial U\partial V}\vspace{2mm}\\
	\dfrac{\partial ^2 S}{\partial U\partial V} & \dfrac{\partial^2 S}{\partial V^2}
	\end{pmatrix}.
\end{align}
Ennek a mátrixnak a stabilitáshoz \emph{negatív definitnek\footnote{\,Egy $\bm M$ mátrix negatív definit, ha minden $\bm r\in\mathbb R^n{\setminus} \bm 0 $ vektorra $\bm r\bm M \bm r<0$ teljesül, ami ekvivalens azzal, hogy az összes sajátértéke negatív.}} kell lennie, miből következik, hogy:
\begin{align}
	\frac{\partial^2 S}{\partial U^2}<0\follows C_V>0,\quad \tn{és}\quad \det \bm H>0\follows \kappa>0,
\end{align}
ahol $\kappa$ újfent a kompresszibilitást jelöli. Ezt hívják \emph{Le Chatelier--Braun\footnote{\,Karl Ferdinand Braun, 1850-1918.}-elvnek}, amikor mindkét feltétel teljesül. Ha $\kappa<0$ lenne, akkor a dugattyú a nagyobb nyomású térrész felé mozdulna el.
