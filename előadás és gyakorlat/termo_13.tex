\section{II.\ főtétel következményei}
\emph{Maxwell-relációk és alkalmazásuk a JT-folyamat és a RM-egyenlet esetén; Euler-összefüggés; Gibbs--Duhem-reláció}

Ismerjük most már a Legendre-transzformációval származtatott leggyakoribb termodinamikai potenciálokat. Egy rövid ideig ugorjunk vissza a (\ref{eq:termo_II_rev}). egyenletig, amikor még nem foglalkoztunk a változó anyagmennyiség lehetőségével. Ekkor a termodinamikai potenciáloknak csak két szabad változójuk van, ebből pedig összesen négy lehetőségünk van, hogy a konjugált párokból 2-2-t kiválasszunk, melyek rendre a belső energia $U(S,V)$, a szabadenergia $F(T,V)$, a Gibbs-potenciál $G(T,\pres)$ és az entalpia $H(S,\pres)$. Kiszámoltuk már az előbbi mennyiségek valamely változójuk szerinti parciális deriváltjait, azonban nézzük meg, hogy mi történik, ha egy termodinamikai potenciált két természetes változója szerint is parciálisan deriválunk. Ekkor alkalmazható a Young\footnote{\,William Henry Young, 1863-1942.}-tétel\footnote{\,Mely a második deriváltak szimmetriatulajdonságát fejezi ki (azaz a Hesse\footnotemark-mátrix szimmetrikus)}, tehát a parciális deriválások sorrendje felcserélhető.
\footnotetext{\,Otto Hesse, 1811-1874.}
A belső energiát deriválva $S$ és $V$ szerint:
\begin{align}
	\pd US\bigg|_{V,n} = T,\quad \pd UV\bigg|_{S,n} = -\pres ,\quad \frac{\partial^2 U}{\partial V\partial S} = \frac{\partial^2 U}{\partial S\partial V} \follows \pd TV\bigg|_{S,n} = - \pd \pres S\bigg|_{V,n}.
\end{align}
Az így származtatott összefüggéseket nevezzük \emph{Maxwell-relációknak}, melyeket \aref{tab:pot_Maxwell}. táblázatban összegeztünk. 
\begin{table}[h!]
\centering
\begin{adjustbox}{width=1.1\textwidth, center=\textwidth}
\begin{tabular}{|c|c|c|c|c|} \hline
termodinamikai potenciál & jele & természetes változói & differenciális egyenlet & Maxwell-reláció\\ \hline\hline
belső energia & $U$ & $S,V$ & $\m dU = T\m dS{-}\pres \m dV$ & $\dfrac{\partial T}{\partial V}\bigg|_S = -\dfrac{\partial \pres}{\partial S}\bigg|_V$\\ \hline
szabadenergia & $F$ & $T,V$ & $\m dF = {-}S\m dT{-}\pres\m dV$ & $\dfrac{\partial S}{\partial V}\bigg|_T = \dfrac{\partial \pres}{\partial T}\bigg|_V$\\ \hline
Gibbs-potenciál & $G$ & $T,\pres$ & $\m dG = {-}S\m dT {+}V\m d\pres$ & $\dfrac{\partial S}{\partial \pres}\bigg|_T = -\dfrac{\partial V}{\partial T}\bigg|_\pres$\\ \hline
entalpia & $H$ & $S,\pres$ & $\m dH = T\m dS {+} V\m d\pres$ & $\dfrac{\partial T}{\partial \pres}\bigg|_S = \dfrac{\partial V}{\partial S}\bigg|_\pres$\\ \hline
\end{tabular}
\end{adjustbox}
\caption{Termodinamikai potenciálokhoz leggyakrabban társított Maxwell-relációk.}
\label{tab:pot_Maxwell}
\end{table}
Megjegyezzük, hogy mivel itt nem foglalkoztunk az anyagmennyiséggel, ezért csak ezt a négy Maxwell-relációt kaptuk meg, ezek a leggyakrabban használtak, de természetesen ha az anyagmennyiséget (vagy a nagykanonikus potenciálnál a kémiai potenciált) is szabad változónak tekintjük, akkor számos másik Maxwell-reláció is származtatható.

Korábban találkoztunk a Robert Mayer-egyenlet és a Joule--Thomson-együttható levezetése során olyan parciális deriváltakkal, melyeket akkor nem tudtunk kiszámolni. A Maxwell-relációk használatát ezeken a példákon keresztül fogjuk bemutatni. Az általános Robert Mayer-egyenlet levezetése során (ld. (\ref{eq:RM}). egyenlet) kihasználtuk, hogy:
\begin{align}
    \pres + \pd UV\bigg|_T = \frac{T\kappa}{\beta}.
\end{align}
Ehhez induljunk ki abból, hogy:
\begin{align}
    U(S,V) \follows \m dU=T\m dS-\pres\m dV,
\end{align}
tehát:
\begin{align}\label{eq:RM_Maxwell}
    \pres + \pd UV\bigg|_T = \pres + T\pd SV\bigg|_T-\pres\underbrace{\pd VV\bigg|_T}_{=1}=T\pd SV\bigg|_T=T\pd \pres T\bigg|_V = - T \frac{\dfrac{\partial V}{\partial \pres}\bigg|_T}{\dfrac{\partial V}{\partial T}\bigg|_\pres} = \frac{TV\kappa}{V\beta}=\frac{T\kappa}{\beta}.
\end{align}
Először felhasználtuk az egyik Maxwell-relációt majd a hármasszabályt, végezetül pedig a térfogati hőtágulási együttható és a kompresszibilitás definícióját (itt most $\kappa$ kivételesen a kompresszibilitást jelöli).

Lássuk be még a Joule--Thomson-együttható levezetésénél megjelenő deriváltat is (ld. (\ref{eq:JT_coeff}). egyenlet). Ehhez felhasználjuk, hogy:
\begin{align}
    H(S,\pres)\follows \m dH=T\m dS+V\m d\pres.
\end{align}
Ezzel a kérdéses derivált:
\begin{align}\label{eq:JT_Maxwell}
    \pd H\pres\bigg|_T = T\pd S\pres\bigg|_T + V\underbrace{\pd \pres\pres\bigg|_T}_{=1} = -T\pd VT\bigg|_\pres + V= -TV\beta +V = V(1-T\beta),
\end{align}
ahol felhasználtunk egy másik Maxwell-relációt, illetve a térfogati hőtágulási együttható definícióját.

Azt érezzük, hogy ha kétszer akkora térfogatú gázt veszünk, akkor a rendszer belső energiája is megnő. Ahhoz, hogy ezt matematikai formába önthessük definiáljuk a \emph{k-ad rendű homogén függvényeket}, melyekre igaz, hogy:
\begin{align}
	f(\lambda x_1,\lambda x_2,\dots,\lambda x_N) = \lambda^k f(x_1,x_2,\dots,x_N).
\end{align}
Ha megvizsgáljuk a belső energiára felírt fundamentális egyenletet (ld. (\ref{eq:U_fund}). egyenlet), és a kifejezésben mindenhol a $\lambda$-szorosát vesszük a természetes változóknak, akkor láthatjuk, hogy a belső energia a természetes változóinak \emph{elsőrendű homogén függvénye}, azaz:
\begin{align}\label{eq:homogen}
	U(\lambda S,\lambda V, \lambda n) = \lambda U(S,V,n).
\end{align}
Deriváljuk le ezt a kifejezést külön-külön a természetes változók szerint:
\begin{align}
	\pd{}{S}&: &\pd US\bigg|_{V,n}\hspace{-3mm}(\lambda S,\lambda V, \lambda n)\cdot \lambda &= \lambda \pd US\bigg|_{V,n}\hspace{-3mm} (S,V,n)\quad \Rightarrow &T(\lambda S,\lambda V, \lambda n) &= T(S,V,n),\\
	\pd{}{V}&: &\pd UV\bigg|_{S,n}\hspace{-3mm}(\lambda S,\lambda V, \lambda n)\cdot \lambda &= \lambda \pd UV\bigg|_{S,n}\hspace{-3mm} (S,V,n)\quad \Rightarrow &\pres(\lambda S,\lambda V, \lambda n) &= \pres(S,V,n),\\
	\pd{}{n}&: &\pd Un\bigg|_{S,V}\hspace{-3mm}(\lambda S,\lambda V, \lambda n)\cdot \lambda &= \lambda \pd Un\bigg|_{S,V}\hspace{-3mm} (S,V,n)\quad \Rightarrow &\mu(\lambda S,\lambda V, \lambda n) &= \mu(S,V,n),
\end{align}
azaz ezek \emph{nulladrendű homogén függvények}, ami azt jelenti pontosan, hogy \emph{intenzív állapotjelzők}.
Deriváljuk most le $\lambda$ paraméter szerint is a (\ref{eq:homogen}). összefüggést:
\begin{align}
\begin{split}
	&\pd US\bigg|_{V,n}\hspace{-3mm}(\lambda S,\lambda V, \lambda n)\cdot S + \pd UV\bigg|_{S,n}\hspace{-3mm}(\lambda S,\lambda V, \lambda n)\cdot V+\pd Un\bigg|_{S,V}\hspace{-3mm}(\lambda S,\lambda V, \lambda n)\cdot n =\\
	&= T(\lambda S,\lambda V, \lambda n)\cdot S-\pres(\lambda S,\lambda V, \lambda n)\cdot V +\mu(\lambda S,\lambda V, \lambda n)\cdot n =\\
	&= T(S,V,n)\cdot S - \pres(S,V,n)\cdot V + \mu(S,V,n)\cdot n = U(S,V,n),
\end{split}
\end{align}
azaz
\begin{align}
	U=TS-\pres V+\mu n,
\end{align}
amit \emph{Euler-összefüggésnek} nevezünk. Így a termodinamikai potenciálfüggvények is átírhatók:
\begin{align}
	U &= TS-\pres V +\mu n\\
	F &= U-TS = -\pres V+\mu n\\
	G &= U-TS+\pres V = \mu n\\
	H &= U+\pres V = TS+\mu n\\
	\Phi &= U-TS-\mu n = -\pres V.
\end{align}
Érdekes tulajdonsághoz juthatunk, ha képezzük az Euler-összefüggés differenciális alakját:
\begin{align}
\begin{split}
	\m dU &= \m d(TS-\pres V+\mu n) = \m d(TS)-\m d(\pres V)+\m d(\mu n) =\\
	&= \underbrace{T\m dS-\pres \m dV+\mu \m dn}_{=\m dU} +S\m dT-V\m d\pres +n\m d\mu \follows 0 = S\m dT-V\m d\pres +n\m d\mu,
\end{split}
\end{align}
amit \emph{Gibbs--Duhem\footnote{\,Pierre Duhem, 1861-1916.}-relációnak} nevezünk.