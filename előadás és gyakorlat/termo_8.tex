\section{A termodinamika II.\ főtétele}
\emph{Reverzibilis/irreverzibilis folyamatok; Kvázi-sztatikusság és reverzibilitás kapcsolata; Példák reverzibilis folyamatokra; II. főtétel: Clausius és Kelvin--Planck-féle megfogalmazás}

A termodinamika I. főtétele kimondja az energiamegmaradást, amiből következik, hogy a részecskék között ható erők konzervatívak. Ez minden esetben teljesül, $\m dU = \delta W + \delta Q$. Tudjuk továbbá, hogy ha két test termikus kölcsönhatásba kerül, akkor a hőmérsékleteik kiegyenlítődnek, mégpedig ha kezdetben a két test hőmérséklete $T_1$ és $T_2$ volt ($T_1<T_2$), akkor a közös hőmérséklet $T_3$-ra $T_1<T_3<T_2$ lesz igaz. Hasonlóan ha két különböző $\pres_1$ és $\pres_2$ nyomású ($\pres_1<\pres_2$) térrészt összenyitunk, akkor a kialakuló egyensúlyi nyomás $\pres_3$ a kezdeti nyomások között fog elhelyezkedni, azaz $\pres_1<\pres_3<\pres_2$. A termodinamika I. főtétele nem tiltja meg, hogy ezek a folyamatok visszafelé is lejátszódjanak, azaz például energiabefektetés nélkül két kezdetben azonos hőmérsékletű test közül az egyik egyszer csak lehűljön, a másik pedig felmelegedjen, azonban ez teljesen ellentmond a fizikai tapasztalatainknak.

Ez alapján két csoportra oszthatjuk a termodinamikai\footnote{\,Igazából ez egy picit félrevezető elnevezés, mert a kurzus során csak egyensúlyi folyamatokkal foglalkozunk, nincs dinamika. Helyesebb lenne talán a ,,termosztatika'' elnevezés, bár ezt nem használják, ezért maradunk mi is a termodinamika kifejezés használatánál.} folyamatainkat, melyeket \emph{reverzibilis} illetve \emph{irreverzibilis folyamatoknak} nevezünk. Alább felsoroljuk az \emph{irreverziblilis} folyamatok néhány jellemzőjét:
\begin{itemize}
    \item a folyamat nem megfordítható olyan értelemben, hogy a folyamatot irányító külső hatásokat fordítva alkalmazva a rendszer és a környezete nem jut vissza eredeti állapotába,
    \item nem kvázisztatikus folyamatok, azaz a folyamat közben nincs egyensúly,
    \item kvázisztatikus rendszerek között valamely állapotjelző kiegyenlítődik (pl. nyomás, hőmérséklet, kémiai potenciál),
    \item súrlódással járó folyamatok is irreverzibilisek, mert a munka hővé alakul,
    \item szintén irreverzibilis folyamat, ha áram folyik ellenálláson, és Joule-hő fejlődik,
    \item a kémiai reakciók jelentős része is irreverzibilis,
    \item maradandó alakváltozás,
    \item illetve ferromágneses anyagok átmágnesezése is ilyen folyamat.
\end{itemize}
Ezzel szemben \emph{reverzibilis folyamatról} beszélünk, amikor:
\begin{itemize}
    \item a folyamat megfordítható,
    \item minden kvázisztatikus (tehát folyamatosan egyensúlyban lévő) folyamat reverzibilis\footnote{\,Úgy tűnhet például, hogy a disszipációval járó folyamatok is kvázisztatikusak, és emiatt irreverzibilisek, azonban vegyük észre, hogy például  súrlódás esetén nincs egyensúly a folyamat közben.}.
\end{itemize}
Azt gondolhatnánk, hogy egy folyamat irreverzibilis, ha a munkavégzés teljes egészében hővé alakul, azonban ez nem igaz, hiszen képzeljük el, hogy a gázt izotermikusan összenyomjuk. Emiatt a belső energia nem változik meg, $\Delta U =0$. A munkavégzés során keletkezett hőt teljes egészében a környezetének adja át, mely ideális gáz esetében:
\begin{align}
    \Delta U =0,\quad W = nRT\ln\z{\frac{V_1}{V_2}},\quad Q = -W = - nRT\ln\z{\frac{V_1}{V_2}}.
\end{align}
Azonban ha visszaengedjük a dugattyút, akkor az ehhez szükséges hőt a környezetéből (külső hőtartály) fogja felvenni a gáz. Habár van hőcsere a környezettel, nem történik kiegyenlítődés, hiszen a hő azonos hőmérsékletű anyagok között áramlik. Tehát ez a folyamat reverzibilis, nincs kiegyenlítődés. Az ilyen folyamatot, amikor a hőcsere két azonos hőmérsékletű test között megy végbe \emph{reverzibilis hőcserének} nevezzük.

Egy \emph{reális} folyamat során azonban - amikor ezt az összenyomás gyorsan csináljuk - az összenyomás során a gáz picit felmelegszik, tágulás során pedig picit lehűl, miközben persze folyamatos a hőcsere a hőtartállyal.

\emph{Termodinamikai körfolyamatnak} nevezzük az olyan termodinamikai állapotváltozásokat, melyek során a rendszer a kezdeti állapotba tér vissza. Ekkor az összes állapotjelző felveszi a kezdeti értéket, miközben a folyamat során hőcsere és munkavégzés is történik. Mi dönti el tehát az előzőek alapján, hogy egy körfolyamat reverzibilis vagy irreverzibilis? Ehhez meg kell vizsgálni, hogy mennyi a teljes hőleadás illetve hőfelvétel, továbbá hogy mi történik a körfolyamatot végző gáz környezetével. Ha a $\pres-V$ diagramot nézzük, akkor a körfolyamat során végzett munka\footnote{\,Emlékezzük vissza, hogy a térfogati munkavégzés $A$ és $B$ pontok között:$$W_{AB} = -\int_A^B p\m dV.$$} a körfolyamat által határolt területtel egyezik meg (hiszen legyen például a körfolyamat csillagszerű tartomány\footnote{\,Egy $\Omega\in\mathbb R^2$ tartományt csillagszerűnek nevezünk, ha létezik olyan $\bm b\in\Omega$ pontja, hogy minden $\bm x \in\Omega$ esetén a $\bm b$ és $\bm x$ pontokat összekötő szakaszt tartalmazza $\Omega$. Ez igazából olyasmi, mintha a tartomány ,,konvex'' lenne mindenütt.}, illetve tartozzanak $A$ és $B$ pontok a legkisebb és legnagyobb térfogathoz. Ekkor az $A\to B$ integrálból kivonódik a $B\to A$ integrál értéke, így ténylegesen csak a közbezárt terület marad eredményül). Ha azonban a $\pres-V$ diagramon ránézünk egy körfolyamatra, akkor arról nem tudjuk egyértelműen eldönteni, hogy az reverzibilis vagy irreverzibilis-e, ugyanis az ábráról nem látszik, hogy az azonos hőmérsékletű rendszerek között történik-e. Ha \emph{végtelenül sok} hőtartályunk lenne (minden pillanatban azonos hőmérséklet), akkor a körfolyamat reverzibilis lenne, azonban általában nem ez a helyzet. Fontos, hogy ez nyomásra is igaz, nem csak hőmérsékletre. Egy körfolyamat akkor lenne reverzibilis, ha a kezdeti állapotba jutunk vissza, amimkor oda- illetve visszafelé is megyünk egy kört.

Az eddigi tapasztalatainkat foglalja össze a \emph{termodinamika II. főtétele}, melynek látni fogjuk, hogy két egymással ekvivalens megfogalmazása van. Az egyik szerint hő nem mehet alacsonyabb hőmérsékletű helyről magasabb hőmérsékletű helyre anélkül, hogy a környezetében valami változás vissza ne maradjon (azaz anélkül, hogy ne végeznénk rajta munkát), ez a \emph{Clausius\footnote{\,Rudolf Clausius, 1822-1888.}-féle megfogalmazás} (1850). Másik megfogalmazás szerint nem konstruálható olyan periodikusan működő gép, mely csupán egyetlen hőtartállyal áll kapcsolatban, és munkát végez. Ez a hőtan II. főtételének Kelvin--Planck-féle megfogalmazása (1851. és 1897. rendre). Az ilyen gép lenne a \emph{másodfajú perpetuum mobile}, mely a II. főtétel szerint nem létezhet. A továbbiakban a körfolyamatokat fogjuk tovább boncolgatni.