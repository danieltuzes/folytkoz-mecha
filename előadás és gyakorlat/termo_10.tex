\section{Entrópia}
\emph{Redukált hő körfolyamatokra; Entrópia bevezetése; Entrópia viselkedése hőmérsékletkiegyenlítődés és a Gay-Lussac-kísérlet során; Irreverzibilitás mikroszkopikus értelmezése; Fluktuációk; Entrópia mikroszkopikus definíciója; I. és II. főtétel egyesített alakja}

Láttuk már korábban a Carnot-folyamat tárgyalásánál (ld. (\ref{eq:Carnot_S}). egyenlet), hogy abban az esetben:
\begin{align}\label{eq:Carnot_S1}
	\frac{Q_{12}}{T_1}+\frac{Q_{34}}{T_2} = 0 \follows \oint \frac{\delta Q}{T} = 0.
\end{align}
Most rakjunk egymás mellé két Carnot-körfolyamatot. A legnagyobb és legkisebb hőtartály hőmérséklete legyen $T_1$ és $T_2$, a közös hőtartály hőmérséklete pedig $T_3$. A felső körfolyamat által $T_1$ hőmérsékletű termosztáttól felvett hő $Q_1$, a körfolyamat munkavégzése $W_1$, az alsó körfolyamat által $T_2$ hőmérsékletű termosztátnak leadott hő $Q_2$, a körfolyamat munkavégzése $W_2$, a $T_3$ hőmérsékletű termosztáttól felvett hő (alsó körfolyamat) és leadott hő (felső körfolyamat) különbsége pedig legyen $\Delta Q$. Ebből látható (\ref{eq:Carnot_S1}). összefüggéshez hasonlóan, hogy a:
\begin{align}
	\begin{array}{rrcl}
{\tn{felső Carnot-folyamatra:}}&0&=&\dfrac{Q_1}{T_1}-\dfrac{Q_2}{T_2}\vspace{1mm}\\
{\tn{alsó Carnot-folyamatra:}}&0&=&\dfrac{Q_2}{T_2}+\dfrac{\Delta Q}{T_2}-\dfrac{Q_3}{T_3}\\
\end{array}
\follows
0 = \frac{Q_1}{T_1}+\frac{\Delta Q}{T_2}-\frac{Q_3}{T_3},
\end{align} 
azaz ahogy Carnot-körfolyamat esetén is már láttuk:
\begin{align}
	\sum_i \frac{Q_i}{T_i}=0 \follows \oint\frac{\delta Q}{T} = 0.
\end{align}
Korábban már az állítottuk, hogy a Carnot-folyamat a létező legnagyobb hatásfokú körfolyamat. Ennek bizonyítására vizsgáljuk meg ennél az egymás mellé helyezett két Carnot-körfolyamatból álló rendszer hatásfokát is. Vezessük be az alábbi jelöléseket:
\begin{align}
	\eta_{12}=1-\frac{T_2}{T_1},\quad \eta_{13}=1-\frac{T_3}{T_1},\quad \eta_{32}=1-\frac{T_2}{T_3}
\end{align}
Mivel $Q_1$ a $T_1$ hőtartálytól felvett hő, ezért a felső Carnot-folyamat munkavégzése $\eta_{13}Q_1$, míg a $T_3$ hőtartálynak leadott hő $\frac{T_3}{T_1}Q_1$, mely tovább adódik az alsó Carnot-folyamatnak. Az alsó Carnot-folyamat továbbá felvesz $\Delta Q$ hőt a $T_3$ hőmérsékletű hőtartálytól is, így ennek a munkavégzése $\eta_{32}\big(\frac{T_3}{T_1}Q_1+\Delta Q\big)$, a $T_2$ hőmérsékletű hőtartálynak leadott hő pedig ezek alapján $\frac{T_2}{T_3}\big(\frac{T_3}{T_1}Q_1+\Delta Q\big)$. Ezek ismeretében a teljes körfolyamat hatásfoka:
\begin{align}
\begin{split}
	\eta &= \frac{-W}{Q_1+\Delta Q} = \frac{-W_1-W_2}{Q_1+\Delta Q} = \frac{Q_1+\Delta Q-\dfrac{T_2}{T_3}\z{\dfrac{T_3}{T_1}Q_1+\Delta Q}}{Q_1+\Delta Q} = \\
	&= \frac{1+x-\dfrac{T_2}{T_1}-\dfrac{T_2}{T_3}x}{1+x} = \frac{\eta_{12}^{\tn{Carnot}}+\eta_{32}^{\tn{Carnot}}x}{1+x} = \eta_{12}^{\tn{Carnot}}\underbrace{\frac{1+\dfrac{\eta_{32}^{\tn{Carnot}}}{\eta_{12}^{\tn{Carnot}}}x}{1+x}}_{<1}< \eta_{12}^{\tn{Carnot}},
\end{split}
\end{align}
ahol bevezettük az $x:=\frac{\Delta Q}{Q_1}$ jelölést. Láthatjuk tehát, hogy a két Carnot-folyamatból álló rendszer hatásfoka valóban kisebb, mint egy Carnot-körfolyamaté. Az állítás végső igazolásához mindösszesen annyi a teendőnk, hogy ezután bármely körfolyamatot izotermákkal és adiabatákkal közelítjük, így sok Carnot-körfolyamat hatásfokából áll össze a teljes körfolyamat hatásfoka. Ha sok hőtartályunk van, akkor a \emph{legnagyobb hatásfokú gép} egy olyan \emph{Carnot-folyamat}, ami a \emph{legnagyobb és a legkisebb hőtartály} között megy végbe. Továbbá mivel a Carnot-folyamat hatásfoka minden anyag esetében ugyanakkora, ez a levezetés tetszőleges anyagra igaz.

Bármilyen körfolyamatot közelíthetünk tehát sok-sok Carnot-körfolyamattal, melyeket összeadva kapjuk (hiszen külön-külön, illetve kettőre is teljesült, így $N$ darabra is), hogy:
\begin{align}
	\oint \frac{\delta Q}{T} = 0,
\end{align}
az integrandusban megjelenő kifejezést pedig elnevezzük \emph{entrópiának}, azaz:
\begin{align}
	\m dS:= \frac{\delta Q}{T}, \quad [S]=\frac JK \follows \oint \m dS = 0,
\end{align}
az entrópia állapotjelző. Ez alapján ha egy $A$ pontban rögzítjük az entrópiát, akkor egy $B$ pontban kiszámolható, mint:
\begin{align}
	S(B) = S(A) + \underbrace{\int_A^B\frac{\delta Q}{T}}_{=\Delta S},\quad\tn{ahol}\quad S(A):=\int_0^A\dfrac{\delta Q}{T},
\end{align}
és $\Delta S$ útvonaltól függetlenül ugyanannyi. Az entrópia definícióját átrendezve kapjuk, hogy:
\begin{align}
	\m dS = \frac{\delta Q}{T} \follows \delta Q = T\m dS,
\end{align}
ami igaz kvázisztatikus folyamatokra\footnote{\,Azért csak kvázisztatikusra igaz, mert a bevezetése során erősen támaszkodtunk a Carnot-folyamatra, ami viszont reverzibilis.}. Mivel $\Delta S = \int \frac{\delta Q}{T}$, ezért $\Delta S \propto n$, azaz arányos az anyagmennyiséggel, tehát extenzív állapotjelző (erre még visszatérünk).
Egy érdekes következtetést vonhatunk le, ha megvizsgáljuk az alábbi rendszert. Legyen két tartályunk, mindkettőben $n$ mólnyi, $C$ mólhőjű anyag van, azonban az egyik $T_1$, a másik $T_2$ hőmérsékleten (legyen $T_1>T_2$). Ha a két tartályt összenyitjuk, akkor a hőmérsékletek kiegyenlítődnek ($T^* = \frac 12(T_1+T_2)$, azonban kérdés, hogy mekkora lesz ekkor $\Delta S$?
\begin{align}
	\Delta S = \underbrace{\int_{T_1}^{T^*}\frac{nC\m dT}{T}}_{<0} + \underbrace{\int_{T_2}^{T^*}\frac{nC\m dT}{T}}_{>0} = nC\kz{\ln\z{\frac{T^*}{T_1}}+\ln\z{\frac{T^*}{T_2}}} = nC\ln\z{\frac{{T^*}^2}{T_1T_2}}.
\end{align}
Felhasználva, hogy $T^* = \frac{T_1+T_2}{2}$:
\begin{align}
	\frac{{T^*}^2}{T_1T_2}=\frac{(T_1{+}T_2)^2}{4T_1T_2} {=} \frac{T_1^2{+}2T_1T_2{+}T_2^2}{4T_1T_2} {=} \frac{T_1^2{-}2T_1T_2{+}T_2^2}{4T_1T_2} {+} \frac{4T_1T_2}{4T_1T_2} {=} 1{+}\underbrace{\frac{(T_1{-}T_2)^2}{4T_1T_2}}_{>0}>1,
\end{align}
aminek a logaritmusa pozitív, azaz $\Delta S>0$, az entrópia nőtt a folyamat során. Ha magából az entrópia definíciójából indulunk ki, akkor tekintsünk két $T_1$ és $T_2$ hőmérsékletű rendszert (legyen megint $T_1>T_2$), melyek között hő áramlik a nagyobb hőmérsékletűtől a kisebb hőmérsékletű felé. Ekkor az entrópia megváltozása:
\begin{align}
	\m dS = \m dS_1+\m dS_2 =\delta Q\z{\rec{T_2}-\rec{T_1}}>0.
\end{align}
A negatív előjel azért jelent meg, mert a $T_1$ hőmérsékletű rendszerből áramlott ki a hő. Általános következmény, hogy ha két test között \emph{hőmérséklet kiegyenlítődés} van, akkor a rendszer teljes \emph{entrópiája nő} (még akkor is, ha az egyes alrendszereké esetleg csökkenhet). Csak akkor nem nő a rendszer összeentrópiája, ha $T_1=T_2$. Clausius 1862-ben ez alapján a \emph{termodinamika II. főtételét} az alábbiakban foglalta össze: zárt rendszer teljes entrópiája nem csökkenhet, az entrópia pedig csak reverzibilis folyamatoknál marad állandó, azaz:
\begin{align}
	\m dS \geq \frac{\delta Q}{T} \follows
\begin{array}{rl}
=&\tn{ha reverzibilis}\\
>&\tn{ha irreverzibilis},
\end{array}
\end{align}
ezt hívják \emph{Clausius-egyenlőtlenségnek}.

Létezik egy ún. hőhalál elmélet, mely szerint sok idő után az entrópia mindenhol kiegyenlítődik, ezért nem lesz hőáramlás, de szerencsére ezt a problémát a gravitáció és az Univerzum tágulása megoldja. Az entrópia ismeretében megfogalmazhatjuk a \emph{termodinamika I. és II. főtételének egyesített alakját}:
\begin{align}
	\m dU = \delta Q + \delta W \leq -\pres\m dV + T\m dS.
\end{align}
Eddig az történt, hogy a $\int_A^B\frac{\delta Q}{T}$ mennyiséget elneveztük entrópia megváltozásának $A$ és $B$ pontok között. Célszerű lenne azonban valamilyen \emph{szemléletesebb jelentést} is társítanunk az entrópiának. Vizsgáljuk meg újra a Gay-Lussac-kísérletet ideális gázzal\footnote{\,Emlékezzünk vissza: van két tartály, az egyikben $\pres_1$ nyomású $V_1$ térfogatú gáz, a másikban pedig $V_1$ térfogatú vákuum van, az egész rendszer pedig egy nagy hőtartályba van behelyezve. A két tartályt kezdetben egy csappal elzárjuk egymástól.}. A korábban bevezetett $\Delta S = \int_A^B\frac{\delta Q}{T}$ összefüggés csak kvázisztatikus folyamatokra volt igaz, azonban amikor a Gay-Lussac-kísérletnél kinyitjuk a csapot a folyamat közel sem kvázisztatikus, ezért erre nem alkalmazhatjuk az entrópia ezen definícióját. Azonban ha az összenyitás után a gázt izotermikusan összenyomjuk az eredeti térfogatra, akkor az entrópiaváltozás pont az eredeti -1-szerese lesz, hiszen:
\begin{align}
	\m dU = \delta Q+\delta W = 0 \follows \delta Q = -\delta W.
\end{align}
Izoterm folyamat során viszont a munkavégzést ki tudjuk számolni:
\begin{align}
	W = -\int_{2V_1}^{V_1}\pres \m dV = -\int_{2V_1}^{V_1}\frac{nRT}{V}\m dV = \int_{V_1}^{2V_1}\frac{nRT}{V}\m dV = nRT\ln 2,
\end{align}
azaz ekkor az entrópiaváltozás:
\begin{align}
	\Delta S = \frac QT = -\frac WT = -nR\ln 2 \follows \Delta S_{\tn{G-L}} = nR\ln 2 = Nk_B\ln2 = k_B\ln\z{2^N},
\end{align}
ahol felhasználtuk a Boltzmann-állandó definícióját. Az \emph{entrópiát statisztikus fizikai módon} is értelmezhetjük. Ehhez legyen két ugyanakkora térfogatú dobozunk (bal és jobb oldali), melyek között rés található. Az egyik dobozban piros, a másikban kék golyók vannak. A dobozokat elkezdjük rázni, emiatt egy idő után a piros és kék golyók összekeverednek. Miért nem tud ekkor magától visszaállni a rendszer az eredeti állapotába? Legyen az összes golyók száma $N$. Jelöljük $\prob$-vel annak a valószínűségét, hogy egy golyó a bal oldali térrészben van. Annak a valószínűsége, hogy $k$ golyó van a bal oldali dobozban és $N-k$ a másikban a binomiális eloszlásnak megfelelően:
\begin{align}
	\mathtt P_k = \frac{N!}{k!(N-k)!}\prob^k (1-\prob)^{N-k},
\end{align}
ahol $\mathtt P$ jelöli az esemény bekövetkeztének a valószínűségét.
A binomiális eloszlás várható értéke $\langle k\rangle = \prob N$. Mivel a két doboz térfogata megegyezik, ezért $\langle k\rangle = \frac N2$. Termodinamikai határesetben $N \to \infty$ és $\prob\to 0$, ekkor a binomiális eloszlás Poisson-eloszlásba\footnote{\,A Poisson-eloszlás:
$$ \mathtt P_k = \frac{\langle k\rangle ^k}{k!} e^{-\langle k \rangle}.$$} megy át, melynél
\begin{align}
	\sigma^2=\langle(k-\langle k \rangle)^2\rangle = \langle k \rangle,
\end{align}
azaz a szórásnégyzet megegyezik a várható értékkel. Mivel makroszkopikusan nagy rendszerekkel foglalkozunk, ezért ez a határeset igaz nálunk is.
Ekkor a bal oldali dobozban lévő részecskék száma:
\begin{align}
N_b = \frac N2 \pm \Delta N,\quad\tn{ahol}\quad \Delta N = \sqrt N \quad\tn{a szórás}. 
\end{align}
Definiáljuk még egy mennyiség relatív szórását is, mint:
\begin{align}
	\frac{\langle (k- \langle k \rangle)^2\rangle}{\langle k\rangle ^2} = \frac{1}{\langle k \rangle},
\end{align}
ahol felhasználtuk a Poisson-eloszlás szórásnégyzetének értékét. Ez az összefüggés megadja a mi esetünkben az eloszlás félértékszélességét is. Ha nagyon sok atomot teszünk egy dobozba, legyen például $2\cdot 10^{24}$ db, akkor a bal oldalon $10^{24}\pm 10^{12}$ db atom lesz, a relatív szórás pedig $10^{-12}$, tehát az eltérés nagyon kicsi. Nagyobb részecskeszám az ideális gáz állapotegyenlete miatt nagyobb nyomást is eredményez, tehát az előbbi számokkal a két oldal közötti nyomás ingadozása is $10^{-12}$. Az atomok 1-1 lehetséges elrendeződését mikroszkopikus állapotnak vagy \emph{mikroállapotnak} nevezzük, jelöljük ezek számát $\Omega$-val. Számos mikroállapot azonban ugyanazt a makroszkopikus állapotot jeleníti meg (például ugyanolyan részecskék esetén ha kettő helyet cserél, akkor a rendszer makroszkopikusan még ugyanúgy viselkedik). Ahhoz a konfigurációhoz, elrendeződéshez csak egyetlen mikroállapot tartozik, amikor az összes atom az egyik térrészben van, míg a teljesen \emph{rendezetlen állapothoz}, amikor a két térrész egyenlő mértékben van kitöltve sokkal több mikroállapot tartozik, ezért ez nagyobb valószínűséggel valósul meg, ez pont az irreverzibilitás. 
Legyen kezdetben az egyik térrész térfogata $V_1$, entrópiája $S_1$, az ebben lévő mikroállapotok száma pedig $\Omega_0$. Amikor összenyitjuk a két dobozt ($2V_1$ térfogat és $S_2$ entrópia), $N$ darab részecskét a két térrész között $2^N$ féle módon helyezhetünk el, illetve egy térrészen belül is $\Omega_0$ lehetőségünk van, így az összes mikroállapotok száma ebben az esetben:
\begin{align}
	\Omega = 2^N\Omega_0.
\end{align}
Bevezetjük a \emph{Boltzmann-entrópia}\footnote{\,Ez az összefüggés van a sírjára is írva.} fogalmát, amivel a későbbiekben kicsit részletesebben is megismerkedünk, mely szerint:
\begin{align}
	S = k_B\ln\Omega.
\end{align}
Ezzel együtt már felírhatjuk az entrópia megváltozását:
\begin{align}
	\Delta S = S_2-S_1 =k_B\ln\big(2^N\Omega_0\big) - k_B\ln\Omega_0 = k_B\ln\big(2^N\big)= Nk_B\ln 2,
\end{align}
ami megegyezik a Gay-Lussac-kísérletből kapott eredménnyel. Végezetül nézzünk meg egy fontos következményt: ha a két részben lévő mikroállapotok száma $\Omega_1$ és $\Omega_2$, akkor mivel a két rendszer egymástól független, az összes lehetséges mikroállapotok száma:
\begin{align}\label{eq:mikroallapotok}
	\Omega = \Omega_1\cdot \Omega_2\follows \ln\Omega = \ln\Omega_1 + \ln\Omega_2 \follows S = S_1+S_2,
\end{align}
azaz az entrópia valóban additív mennyiség, azaz extenzív állapotjelző.
Az eddigiek tükrében már érthető, hogy miért szokás az entrópiát a \emph{rendezetlenség mértékének} is nevezni. A rendezett (szabály szerinti) makroállapothoz biztosan kevesebb mikroállapot tartozik, mint egy rendezetlenebb állapothoz. Ezért biztosan keveredni fognak a részecskék, a rendezetlenségük növekszik, amíg el nem érik a legrendezetlenebb állapotot. Az entrópia jellemzi a rendszer \emph{információtartalmát is}, ugyanis megadja, hogy mennyi plusz információra van szükségünk ahhoz, hogy az ismert entrópiájú rendszer állapotát leírjuk\footnote{\,Érdemes utánajárni a Shannon-entrópiának, mely hozzárendelhető az információelmélet egy tetszőleges véletlen változójához. A Shannon-entrópia kiszámolható, mint:
$$S(x_1,x_2,\dots x_N):=-\sum_{i=1}^N \prob_i\log_2(\prob_i),$$
ahol $\prob_i$ az $x_i$ bekövetkezésének valószínűsége.}. Ha a részecskék bárhol lehetnek, akkor több információ kell a leírásához.

Az entrópia ismeretével egy rendszer állapotát már $T-S$ diagramon is tudjuk ábrázolni. Carnot-folyamatoknál ez azt jelenti, hogy mivel a folyamat izotermákon és adiabatákon (izentropikus szakasz) zajlik, a $T-S$ síkon téglalapot kapunk. Legyen $T_1>T_2$ és $S_2>S_1$. Ekkor felírhatjuk, hogy:
\begin{align}
\delta Q = T\m dS \follows \Delta Q = \int_1^2 \delta Q = \int_1^2 T \m dS.
\end{align}
A hatásfok meghatározható ekkor a definíciója szerint, mint a körfolyamat által határolt téglalap területe, eloszva a $T_1$ hőmérsékletű izoterma alatti teljes területtel, azaz:
\begin{align}
\eta = \frac{(T_1-T_2)(S_2-S_1)}{T_1(S_2-S_1)} = \frac{T_1-T_2}{T_1},
\end{align}
ami megegyezik a korábban kapott összefüggéssel.